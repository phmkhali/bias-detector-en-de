\documentclass[11pt,twoside=off,
bibliography=totoc,listof=totoc,appendixprefix,paper=a4,headings=small]{scrbook} % 'twoside=on' für Druckversion oder 'twoside=off' für Onlineversion

%
% Packages
% -----------------------------------
\usepackage[
  paper=a4paper,
  left=12.5mm,
  right=30mm,
  top=25mm,
  bottom=50mm,
  bindingoffset=10mm]{geometry}		% Seitenränder und Bindungskorrektur einstellen

\usepackage[utf8]{inputenc}         % Umlaute im Text

\usepackage[hyphens]{url}
\usepackage[
  backend=biber,       
  style=authoryear,    
  natbib=true,        
  maxcitenames=2,
  mincitenames=1,
  sorting=nyt,         
  giveninits=true
]{biblatex}
\addbibresource{Literatur/refs.bib}  % Make sure this path is correct
\usepackage{pdfpages}
\usepackage{chngcntr}
\counterwithout{figure}{chapter}
\counterwithout{table}{chapter}
\usepackage[titletoc]{appendix}

\usepackage[T1]{fontenc}
\usepackage{lmodern}				% Schriftfamilie
\setkomafont{disposition}{\bfseries}
\usepackage{listings}
\lstset{
  basicstyle=\ttfamily\small,
  breaklines=true,
  columns=fullflexible
}
\usepackage{graphicx} 				% Grafiken einfügen (pdf,png - aber jpg vermeiden)
\usepackage{subcaption}
\graphicspath{{./Bilder/}}          % Pfad zu den Bildern
\usepackage{enumitem}
\usepackage{tikz}
\usetikzlibrary{shapes, arrows.meta, positioning, calc, shapes.geometric, shadows}
\usepackage{tcolorbox}
\tcbuselibrary{listings, breakable}

\usepackage[final,nopatch=footnote]{microtype}

\usepackage{url}					% URL's formatieren (z.B. in Literatur) 
\usepackage[colorlinks,linkcolor=black,citecolor=black,urlcolor=blue]{hyperref} 				% für Hyperlinks in PDF-Dokumenten   
  
\usepackage{tabularx} 				% bessere Gestaltung von Tabellen
\usepackage{longtable} 		
\usepackage{multicol}				
\usepackage{multirow}
\usepackage{booktabs}
\usepackage{float}
\usepackage{tabularx}
\usepackage{forest}
		
\usepackage[active]{srcltx}

\usepackage{listings}				% Algorithmen

\usepackage{mdwlist}				% Listen

\usepackage{setspace} 				% Zeileneinstellung
\newtheorem{mydef}{Merksatz}  		% Falls Beispiele, Merksätze m. fortl. Nr. gebr. werden
\newtheorem{bsp}{Beispiel}

\usepackage{lscape}					% zum Rotieren von Seiten

\usepackage{amsmath}				% zum Schreiben von mathematischen Formeln
\usepackage{amssymb}   
\usepackage{amsfonts} 

\usepackage{calc}

\usepackage{footnote}				% Fußnoten
\usepackage{tablefootnote}			% Fußnoten in Tabellen
\usepackage{makecell} 

%\clubpenalty = 10000
%\widowpenalty = 10000 \displaywidowpenalty = 10000

\hyphenation{voll-st\"andigen}		% Worttrennungen global definieren

\setcounter{tocdepth}{2}			% Ebenen, die im Inhaltsverzeichnis angezeigt werden

% Document
% -----------------------------------
\begin{document}
%TC:ignore
\frontmatter 
    % Titelseite soll keine Kopf oder Fußzeile haben
\thispagestyle{empty}

% Alle Elemente sollen zentriert sein
\begin{center}

\vspace*{-10mm}
\includegraphics[width=0.45\textwidth]{Hochschule_für_Wirtschaft_und_Recht_Berlin_logo.svg.png}\\[10mm]

Berlin School of Economics and Law\\
Department I - Business and Economics\\[15mm]

{\Large \textbf{Detecting Gender Bias in}}\\ 
\vspace*{2mm}
{\Large \textbf{English-German Translations}}\\ 
\vspace*{2mm}
{\Large \textbf{using Natural Language Processing}}\\

\vspace*{\fill} 

{\LARGE {Bachelor's Thesis}}\\ 

\vspace*{\fill} 

for the attainment of the academic degree Bachelor of Science (B.Sc.)\\ \vspace*{1.5mm} 
in the study program\\\vspace*{1.5mm}
\textbf{Information Systems Management}\\\vspace*{1.5mm}


\vspace*{\fill} 

% Name des/der Autors/Autoren
{\Large Submitted by Khanh Linh Pham}\\[15mm]

\vspace*{\fill} 

% Gutachter, Kontaktdaten und Abgabetermin
\begin{flushleft}
\begin{tabbing}
Main Supervisor:\hspace{1.6cm} \= Prof. Dr. Diana Hristova \\
Secondary Supervisor:\> Prof. Dr. Markus Schaal \\[4mm]
Semester:\> Summer Semester 2025\\
Matriculation no.:\> 77211916753\\
Email:\> klpham04@gmail.com\\[8mm]
\textbf{Date of Submission:} \> \textbf{September 01, 2025}\\
\end{tabbing}
\end{flushleft}

\end{center}

\clearpage{\pagestyle{empty}\cleardoublepage} 			% Titelblatt
    \clearpage{\pagestyle{empty}\cleardoublepage}
    \thispagestyle{empty}


\vspace*{1cm}

\begin{center}
    \textbf{Abstract}
\end{center}

\vspace*{1cm}

\noindent 
Gender bias in English–German Machine Translation often appears in forms such as generic masculine defaulting and occupation stereotyping. These biases can perpetuate unequal representations and feed back into future translation models, reinforcing biased outputs in society. This thesis examines how accurately multilingual BERT (mBERT) can detect such bias. The model was fine-tuned on limited datasets with varying annotation quality, which caused its main limitations. The classifier occasionally (1) misclassifies German gender-fair language forms as biased, (2) fails to detect bias in political and government terms, (3) fails to recognize semantically gendered words as unbiased, (4) is sensitive to punctuation and capitalization, and (5) struggles with sentences that contain both neutral and gendered subjects. Despite these gaps, the model achieved an F1 score of 0.966 and proves effective for core bias cases. It reached 84.6\% accuracy on a small handcrafted evaluation dataset with practical sentences like job postings and edge cases. As an intermediary step, the work offers a trained model, sufficiently effective for practical bias detection, and an application that make biased translations visible while indicating areas for further investigation and improvement. The code is available at \url{https://github.com/phmkhali/bias-detector-en-de}.
    \newpage
    
    \clearpage{\pagestyle{empty}\cleardoublepage}
    \onehalfspacing                  	% Zeilenabstand ab hier 1.5 zeilig
    \tableofcontents 					% Inhaltsverzeichnis
    \clearpage{\pagestyle{empty}\cleardoublepage} 
    
    \listoffigures 					 	% Abbildungsverzeichnis
    \clearpage{\pagestyle{empty}\cleardoublepage}
    
    \listoftables						% Tabellenverzeichnis
    \clearpage{\pagestyle{empty}\cleardoublepage}

    \chapter*{List of Abbreviations}  
\addcontentsline{toc}{chapter}{List of Abbreviations} 
\begin{description}[align=left,labelwidth=3cm]
  \item[EN-DE] English-to-German
  \item[GFL] Gender-Fair Language
  \item[mBERT] Multilingual BERT
  \item[MT] Machine Translation
  \item[NLP] Natural Language Processing
  \item[NMT] Neural Machine Translation
\end{description}

    \clearpage{\pagestyle{empty}\cleardoublepage}
%TC:endignore

% -----------------------------------
%TC:include
\mainmatter 							
    \chapter{Introduction}
Machine Translation (MT) helps millions of people communicate across languages, in daily life and in areas like healthcare, law, and business \citep{kapplAreAllSpanish2025}. Services like Google Translate handle over 200 million users every day \citep{pratesAssessingGenderBias2019,shresthaExploringGenderBiases2022}. It is a fast-growing market. A report by \citet{skyquestMachineTranslationMT2025} valued it at 980 million USD in 2023, with projections reaching 2.78 billion USD. New and more advanced translation models keep appearing, and many of them are free to use. As a result, MT tools are now used to translate large volumes of content across domains.

With this widespread use, the output of MT systems increasingly shapes how people receive and interpret information. But automatic translations are not neutral. There is growing concern about the social effects of biased translations. One key issue is gender bias. MT systems are often trained on large datasets that reflect social norms and stereotypes. If the data contains gender bias, the system will likely reproduce it \citep{choMeasuringGenderBias2019,soundararajanInvestigatingGenderBias2024,smacchiaDoesAIReflect2024}.

A common case is the use of gendered terms in translations of gender-neutral input. For example, the English sentence “The nurse is hard-working” does not say anything about gender. But a translation system may render it in German as “Die Krankenschwester ist fleißig,” which uses the explicitly feminine term \textit{Krankenschwester}. Similarly, “The surgeon is hard-working” may become “Der Chirurg ist fleißig,” using the masculine form \textit{Chirurg}. These choices add gendered assumptions that were not present in the original. Such patterns are not just technical side effects. They can reinforce stereotypes, especially when they appear in job ads, reports, or other public texts.


\section{Motivation}

\subsection{Social and Ethical Importance of Addressing Gender Bias}\label{section:social_and_ethical_importance_of_addressing}
Academia has come to the consensus that MT systems do default to male pronouns when gender in the source sentence is ambiguous \citep{pratesAssessingGenderBias2019,choMeasuringGenderBias2019,rescignoGenderBiasMachine2023}.  In addition, translations often reflect traditional roles, like associating “nurse” with women and “surgeon” with men. This can affect people’s perceptions of jobs and reinforce gender roles.

When used in formal contexts like job descriptions or reference letters, biased translations can shape how a candidate is perceived. If a system always assigns male pronouns to leadership roles and female terms to caregiving roles, it may disadvantage those who do not match those stereotypes \citep{bolukbasiManComputerProgrammer2016}. This is not just a personal issue. It can reduce diversity and go against international standards. Organizations like the United Nations, UNESCO, and the European Union stress the importance of gender equality and inclusive language, making gender equality one of the 17 Sustainable Development Goals for 2030 \citep{sczesnyCanGenderFairLanguage2016,unitednationsAchieveGenderEquality2023}. 

Language also shapes thought. Research shows that readers often interpret masculine forms as male-specific, even if they are supposed to be generic \citep{sczesnyCanGenderFairLanguage2016}. Inclusive forms are more common in official documents, less so in everyday language. However, exposure matters. Frequent use of fair language makes it feel more normal. Detecting and addressing bias in MT can support this shift.

\subsection{Why Detection Systems Are Needed}

Current research on this topic tends to focus more on the quantitative measurement of gender bias \citep{rescignoGenderBiasMachine2023,barclayInvestigatingMarkersDrivers2024a,smacchiaDoesAIReflect2024}. Common methods include counting gendered forms in outputs and comparing them to demographic baselines or human expectations \citep{rescignoGenderBiasMachine2023,pratesAssessingGenderBias2019,savoldiWhatHarmQuantifying2024}. These are useful, but they do not help users identify specific biased translations in real-time. Evaluations are not enough for accountability. 

Other domains, like facial recognition, have already seen progress in active bias detection. For example, \citet{schwemmerDiagnosingGenderBias2020} showed that systems tend to label women more accurately if they match stereotypical appearances (e.g., long hair). Some models even linked female images to words like “kitchen” or “cake” based on bias patterns in training data. For MT, a detection layer is still missing. Without such tools, biased translations are likely to spread unnoticed. A detection system could flag potential bias in real time, improving transparency and encouraging more careful use.

\section{Problem Statement and Research Questions}
\textbf{DRAFT NEED TO REWRITE AFTER IMPLEMENTATION}
This thesis focuses on gender bias in English-to-German (EN-DE) MT. This language pair is widely used in research, with many open datasets and high-quality models available. It also involves a grammatical shift: English has limited gender marking, while German assigns gender to many nouns and pronouns. This structural difference makes gender bias more visible and easier to study in the translation outputs.

The core problem boils down to the significant bias towards the masculine form in EN-DE MTs, sometimes consituting 93-96\% of translations for isolated words \citep{lardelliBuildingBridgesDataset2024}. These outputs often reflect social stereotypes rather than objective translations, yet current systems offer no mechanism to detect or signal when such bias occurs \citep{rescignoGenderBiasMachine2023}. To address this, this thesis deploys a blackbox approach to explore how fine-tuning a pre-trained multilingual BERT model can help detect gender bias in MT outputs. The model takes an input sentence and its corresponding German translation and predicts whether the translation introduces gender bias. 

The translation system used is \href{https://github.com/Helsinki-NLP/Opus-MT?tab=readme-ov-file}{Opus-MT}, an open-source neural MT model. It is widely used in research, supports EN-DE translation, and is trained on real-world corpora, making it suitable for studying translation bias \citep{tiedemannOPUSMTBuildingOpen2020}. Translations are then passed through BERT, trained on a dataset I have constructed by combining and adapting several existing datasets from other researchers. The classifier is lightweight and efficient, aiming for transparent behavior and easy integration into other tools \citep{devlinBERTPretrainingDeep2019}. The final tool highlights biased parts in a simple web demo. The goal is not a perfect classifier but a working prototype that shows how such detection could be integrated into translation workflows.

The main research question is therefore: \textbf{"How can a NLP-based binary classification model detect gender bias in English-German translations?"}. 

\section{Scope}

\textbf{WRITE AFTER IMPLEMENTATION PART}
This thesis focuses only on EN-DE MT. Other language pairs are out of scope.

\section{Limitations}
\textbf{WRITE AFTER IMPLEMENTATION PART}
It becomes especially difficult to detect when sentences contain multiple subjects, indirect references, or ambiguous pronouns. For example, as \citet{barclayInvestigatingMarkersDrivers2024a} explain, the sentence “He went to see her mother” clearly implies three people, while “He went to see his mother” could refer to either two or three. These types of structures introduce ambiguity that makes annotation and evaluation much harder. Creating a dataset that captures such linguistic complexity would require significant effort and careful control of variables. One broader limitation in building datasets for complex scenarios with multiple subjects is the difficulty of isolating the influence of each gendered entity \citep{lardelliBuildingBridgesDataset2024}. When working with natural language sources, it becomes hard to tell what caused the bias in the translation. Because of this, the focus of this thesis is on simpler sentence structures with a single subject. This makes it easier to identify and explain bias patterns. It also fits the intended use case: translating business texts like job advertisements or reports, which rarely involve multiple nested clauses or ambiguous pronouns.
 

\section{Overview of Chapters}
\textbf{WRITE AFTER IMPLEMENTATION PART}

    \clearpage{\pagestyle{empty}\cleardoublepage}		
    \chapter{Theoretical Background}
This chapter introduces the fundamental concepts used throughout this thesis. It explains key definitions and presents an overview of BERT, focusing on the features necessary for building the detection system.

% --------------------------------------------------------------------------------

\section{Definitions}
\subsection{Natural Language Processing and Machine Translation}
    \textbf{NLP} enables machine systems to process human language. The goal is to mimic and understand it as fluently as possible \parencite{smacchiaDoesAIReflect2024,ullmannGenderBiasMachine2022}. Common applications are chatbots, translation tools, speech recognition, and image captioning. \textbf{MT} is a direct application of NLP. It performs automatic translation of text from one language to another \parencite{linMachineTranslationAcademic2009}. Over time, MT systems have developed from rule-based approaches, which depend on hand-crafted grammar rules or aligned sentence data, into more adaptable neural models \parencite{chakravarthiSurveyOrthographicInformation2021}.

    Most modern systems, such as Google Translate and DeepL, rely on neural machine translation (NMT) \parencite{wuGooglesNeuralMachine2016,deeplHowDoesDeepL2021}. These models are trained on large collections of translated texts. They learn to represent the meaning of entire sentences as mathematical structures, enabling more fluent and accurate translations. Unlike earlier approaches, NMT systems take the full sentence context into account, which helps reduce errors and improves the handling of ambiguous or idiomatic language \parencite{wuGooglesNeuralMachine2016}. Throughout this work, all MT systems referenced or applied are neural models.

\subsection{Bias and its Manifestations}
\label{subsection:manifestations_of_gb}
    Bias refers to a tendency to favour or disadvantage certain individuals or groups based on preconceived ideas. It often comes from stereotypes, which are fixed and oversimplified ideas about a social group. In short, stereotypes shape assumptions, while bias influences actual behavior and treatment. Bias takes many forms and can be based on characteristics such as age, disability, gender, ethnicity, religion, or sexual orientation \parencite{ullmannGenderBiasMachine2022}. These biases frequently originate from longstanding cultural and historical beliefs about the expected behavior of different groups. This thesis focuses specifically on gender bias, which is particularly prominent in MT due to the influence of gendered language. Elements such as gendered terms, occupational roles, and grammatical patterns can affect translations and often perpetuate stereotypes because language is closely tied to our thoughts and beliefs. Drawing on key studies that examine gender bias in EN-DE MT \parencite{ullmannGenderBiasMachine2022,rescignoGenderBiasMachine2023,lardelliBuildingBridgesDataset2024,kapplAreAllSpanish2025}, such bias typically manifests in the following forms:

    \subsubsection{Defaulting to Masculine Forms}
        In both singular and plural contexts, the \textit{generic masculine} uses the masculine grammatical gender as the default.
        For example, the sentence "Die Studenten sind im Hörsaal" (The students are in the lecture hall) uses the masculine plural form to refer to a group of students regardless of their gender. It is commonly used in spoken German and other gendered languages \parencite{lardelliBuildingBridgesDataset2024,schmitzGermanAllProfessors2022}, although research has consistently shown that the generic masculine creates a male bias in mental representations, leading readers or listeners to think more of male than female examples \parencite{sczesnyCanGenderFairLanguage2016}. 

    \subsubsection{Reinforcement of Stereotypes}
        The gendered language patterns discussed earlier reflect broader social beliefs about men's and women's roles in work and family life. Although many of these roles no longer reflect reality, they continue to shape judgments about people's abilities and personalities. This often leads to correspondence bias, where traits are inferred based on behavior or circumstances \parencite{godsilEffectsGenderRoles2016}. Such stereotypes are reinforced by media, including television and advertising, and influence how language is used and understood. One common result of this is stereotypical job associations. People often link professions like doctors or pilots with he/him pronouns, and professions like nurses or flight attendants with she/her pronouns \parencite{shresthaExploringGenderBiases2022}. \textcite{pratesAssessingGenderBias2019} also found clear patterns in how gender is associated with certain traits. Adjectives like "shy," "happy," "kind," and "ashamed" are often linked to women, while words like "arrogant," "cruel," and "guilty" are more often linked to men. 

  
 \subsection{Gender Bias} \label{subsection:definition_gb}
    A clear definition of gender bias in MT does not exist, nor is there a standard method to identify indicative features in text \parencite{barclayInvestigatingMarkersDrivers2024a}. This leads this study to use a simple rule-based definition to determine when a translation of a sentence is gender biased.

        \begin{itemize}
        \item A gender-ambiguous subject in the source text is translated with a gendered term, often by defaulting to the generic masculine (e.g., doctor → Arzt) or reflecting stereotypical gender roles (e.g., nurse → Krankenschwester).
        \item A gendered subject in the source text is assigned an incorrect gender in the translation, leading to semantic inconsistency (e.g., my mother is an engineer → meine Mutter ist ein Ingenieur).
        \end{itemize}

    \noindent This does not mean that all other cases are truly "unbiased". Anything that does not fall under these two cases will be referred to as "neutral". This includes, but is not limited to:

        \begin{itemize}
        \item Sentences with no gendered terms, like "The weather is nice".
        \item Accurate translations of gendered input, like "The woman is a coder" → "Die Frau ist eine Programmiererin".
        \item The use of gender-fair alternatives (see \autoref{subsection:german_gfl}).
        \end{itemize}

    \begin{table}[htb]
    \centering
    \begin{tabularx}{\linewidth}{X | X}
        \toprule
        \textbf{Biased Translation} & \textbf{Neutral/Fair Translation} \\
        \midrule
        Gender-ambiguous source is translated with a gendered term. & 
        Gender ambiguity is preserved in the translation. \\
        \addlinespace[0.5em]
        Gendered subject is assigned an incorrect gender. & 
        Gender in the translation matches the gendered subject. \\
        \addlinespace[0.5em]
        \multicolumn{1}{c|}{—} & 
        Use of gender-fair language alternatives (see \autoref{subsection:german_gfl}). \\
        \bottomrule
    \end{tabularx}
    \caption[Summary of gender bias scenarios in translation]{Summary of gender bias scenarios in translation (original compilation)}
    \label{tab:overview_bias_neutral}
    \end{table}

    \subsection{Binary Classification in Natural Language Processing}
    Binary classification means sorting items into two clear groups. It is the most common task in Machine Leaning (ML) and is frequently found in every day life, such as automatically flitering e-mails as "spam" or "not spam" \parencite{quemyBinaryClassificationUnstructured2019} or deciding whether a transaction is "fraudulent" or "legitimate". For instance, a spam filter uses previously labelled e-mails to learn relevant patterns by looking at specific keywords or sender information, and builds a model that applies these patterns to classify new messages. This thesis tries to label a translation as either "biased" or "neutral". For example, the translation "The nurse is kind" → "Die Krankenschwester ist nett" would be labelled as biased, whereas a gender-neutral translation such as "Die Pflegekraft ist nett" would be labelled as neutral. While it is possible to extend the classification beyond two categories to distinguish types of bias or include "gender-fair" labels, doing so would require substantially more data and training. Given the practical aim of this work, the simpler binary approach is more suitable.

    % --------------------------------------------------------------------------------

\section{BERT}
BERT, which stands for Bidirectional Encoder Representations from Transformers, is a language model that was introduced by Google in 2018 \parencite{devlinBERTPretrainingDeep2019}. After pre-training, it can be adapted to various NLP tasks, such as text classification, sentiment analysis, or question answering, by adding a simple output layer and fine-tuning on task-specific data. This output layer produces the final prediction, for example by assigning a label to a piece of text, without requiring major changes to the original architecture. BERT's strong language understanding makes it well suited for binary classification tasks. There are multiple variants of the original BERT model. It was originally released in two sizes: \texttt{BERT-Base} and \texttt{BERT-Large}, which differ in the number of layers, attention heads, and overall model capacity \parencite{devlinBERTPretrainingDeep2019}. Since then, many other versions have been developed. Most of them modify either BERT's pre-training objectives or the underlying Transformer architecture \parencite{libovickyHowLanguageNeutralMultilingual2019}.

\subsection{Transformer Architecture} \label{subsection:transformer_arch}
    BERT is built on the transformer architecture. It is a type of neural network for language tasks, where layers learn patterns from language data \parencite{phuongFormalAlgorithmsTransformers2022}. BERT uses a mechanism called \textit{self-attention}, which allows the model to look at all words in a sentence at the same time and capture how they depend on each other. Unlike traditional methods such as Recurrent Neural Networks (RNNs) that process input step by step \parencite{xiaoIntroductionTransformersNLP2023}, self-attention captures global dependencies and contextual relationships more accurately, producing "context-aware" representations. 

    The transformer architecture consists of two main components: the encoder and the decoder. The encoder's job is to read the input sentence and turn it into a series of vectors the model can understand. Each vector is a list of numbers representing the meaning and structure of each word \parencite{xiaoIntroductionTransformersNLP2023}. The encoder works as follows (see \autoref{fig:transformer_architecture}):

    \begin{enumerate}
        \item It receives input embeddings, which represent the words, and positional encodings, which tell the model the order of the words.
        
        \item The data then passes through several identical layers. Each layer has two main components. Each of these is followed by an \textbf{Add \& Layer Norm} step, which helps stabilize and preserve useful information:
        \begin{enumerate}[label=\alph*.]
            \item \textbf{Multi-head self-attention} runs several self-attention processes in parallel. Each attention head focuses on different details to help the model understand the sentence better.
            \item A \textbf{Feed-forward network} processes each word vector separately, refining the information like a small filter.
    \end{enumerate}
    
    \item Each layer builds on the output of the previous one, helping the model form more complex and abstract ideas about the input sentence.
    
    \item Finally, the encoder outputs a sequence of \textit{hidden states}. These are continuous vector representations for each input token. They encode contextual information from the entire sentence. For example, in the sentence "The cat sat on the mat," the vector for "cat" reflects its relationship to words like "sat" and "mat."
\end{enumerate}

\begin{figure}[ht]
    \centering
	\includegraphics[width=\textwidth,height=0.45\textheight,keepaspectratio]{transformer_architecture.png}	
        \caption[Transformer encoder-decoder architecture overview]{Transformer encoder-decoder architecture. The encoder (left) processes input tokens \(x_1,\dots,x_m\) through: (1) a self-attention layer for contextual relationships, (2) a feed-forward network for feature transformation, and (3) residual connections with layer normalization. The decoder (right) generates outputs by attending to both the encoder's representations and its previous outputs ($y_0$ to $y_{n-1}$), producing the next-token probability distribution. Figure and description adapted from \textcite{xiaoIntroductionTransformersNLP2023}, p. 6}
    \label{fig:transformer_architecture}
\end{figure}

The decoder generates the output sentence one word at a time by using the information from the encoder \parencite{xiaoIntroductionTransformersNLP2023}. However, since BERT uses only an encoder-only architecture (see \autoref{fig:bert_arch}), the decoder is not relevant for this work and is therefore excluded from the discussion.

\subsection{Multilingual BERT}
    In this thesis, the model used is multilingual BERT \textbf{\href{https://huggingface.co/google-bert/bert-base-multilingual-cased}{(\texttt{mBERT})}} \parencite{devlinBERTPretrainingDeep2019}. \texttt{mBERT} has the same architecture as \texttt{BERT-Base} but was pretrained on Wikipedia data from 104 languages, including English and German. The model does not receive any explicit signal about which language it is processing. As in the aforementioned "The cat sat on the mat" example, suppose mBERT sees both "The cat sits on the mat" in English and "Die Katze sitzt auf der Matte" in German during training. The model is never told which language a sentence is in; there is no signal saying "this is English" or "this is German." By seeing many such examples, mBERT learns that words like cat and Katze often occupy similar positions in sentences with similar meanings. This allows it to recognize that they play the same role, even without explicit language labels. Its multilingual ability therefore emerges from patterns shared across languages, enabling internal representations that support tasks in multiple languages \parencite{piresHowMultilingualMultilingual2019}. 
    
    \texttt{mBERT} was chosen because it offers a good balance between language coverage, model size, and training efficiency. Monolingual models like \href{https://huggingface.co/google-bert/bert-base-german-cased}{\texttt{German BERT}} do not support English input. Larger multilingual models, such as \href{https://huggingface.co/docs/transformers/en/model_doc/xlm-roberta}{\texttt{XLM-RoBERTa}}, require more computational resources and training time, which was not feasible here. This makes \texttt{mBERT} a practical choice for handling both languages within limited resources.

\begin{figure}
    \centering
	\includegraphics[width=\textwidth,height=0.45\textheight,keepaspectratio]{BERT_architecture.png}	
    \caption[BERT's encoder-only architecture]{BERT's encoder-only architecture Figure by \textcite{smithCompleteGuideBERT2024}}
    \label{fig:bert_arch}
\end{figure}

\subsection{Tokenization}
Before \texttt{mBERT} can process any text, the input must be converted into a format the model can understand. To achieve that, \texttt{mBERT} splits words or subword units into \textit{tokens}. This process is called tokenization\footnote{This tokenization process applies to both BERT and mBERT.}. It uses the WordPiece algorithm with a shared vocabulary of 110,000 tokens \parencite{devlinMultilingualBERTGitHub2018}. To balance the training data, languages with large Wikipedia corpora are downsampled, meaning fewer examples are used, while those with fewer resources are oversampled, meaning some examples are repeated to increase their presence. Pre-processing is the same for all supported languages: (1) converting text to lowercase and removing accents, (2) splitting punctuation, and (3) tokenizing based on whitespace. Removing accents helps reduce the vocabulary size, even though it can introduce ambiguity in languages where accents carry meaning. This trade-off is accepted because \texttt{mBERT's} contextual embeddings usually resolve such ambiguities during training and inference. In addition to tokenizing words and subwords, \texttt{mBERT} relies on \textit{special tokens} to provide structural information. These tokens, such as [CLS] for the start of a sentence or [SEP] to separate segments, are not real words but placeholders that help the model understand the role of different parts of the input. They work together with the tokenized embeddings to give the model a complete representation of the text.

\noindent In this work, each input combines an English source sentence and its German translation as:

\begin{quote}
    \texttt{[CLS] english sentence [SEP] german translation [SEP]}
\end{quote}

\begin{quote}
\texttt{[CLS] the nurse is kind [SEP] die krankenschwester ist nett [SEP]}
\end{quote}

\subsection{Fine-Tuning}
    Fine-tuning adjusts the base model for a specific task, in this case, detecting gender bias in translations. To do so, a new labelled dataset is used to continue training the model, allowing it to adapt its weights to task-specific patterns. In the context of this thesis, adapting the weights means that the model learns to recognize patterns in translations that indicate biased or neutral gender representations.\footnote{This fine-tuning process applies to both BERT and mBERT.} A \textit{classification head} is an additional layer added to the top of the model to turn its general language understanding into task-specific predictions. It usually consists of a \textit{linear layer}, which transforms the model's output into a set of scores, followed by a \textit{softmax function}, which converts these scores into probabilities for each class. Here, the classification head uses the final hidden state of the \texttt{[CLS]} as the input. The linear layer maps this vector to two values (biased or not biased), and the softmax function outputs the probability for each class.\footnote{The following formulas are adapted from \textcite{devlinBERTPretrainingDeep2019} and \textcite{xiaoIntroductionTransformersNLP2023}}

    \[
    z = Wx + b 
    \]

   \(x\) is the \texttt{[CLS]} embedding, \(W\) is the weight matrix, and \(b\) is the bias vector. Both \(W\) and \(b\) are parameters learned during training to help map \texttt{mBERT's} output to the task labels. This changes the output into two numbers (logits), one for each class: biased or neutral. Then, the softmax function turns these numbers into probabilities \parencite{devlinBERTPretrainingDeep2019,xiaoIntroductionTransformersNLP2023}. Short for "soft maximum," it maps raw scores to a probability distribution, emphasizing the highest values while still giving smaller ones some weight.

    \[
    \text{softmax}(z_i) = \frac{e^{z_i}}{\sum_{j=1}^{K} e^{z_j}}
    \]
    
    Each logit \( z_i \) is exponentiated to ensure positivity. The result is then normalized by dividing by the sum of all exponentials, producing the probability distributions. \( K \) is the number of possible classes. The class with the highest probability is selected as the model's prediction. For example, suppose the model outputs logits $[1.5, 0.5]$ for biased and neutral respectively. Exponentiating gives $[e^{1.5}, e^{0.5}] \approx [4.48, 1.65]$. Normalizing by the sum $4.48 + 1.65 = 6.13$ gives probabilities $[0.73, 0.27]$. The model would predict biased since it has the higher probability, with a confidence of 73\%.

    
\subsection{Key Hyperparameters} \label{subsection:hyperparameters_explained}
    Fine-tuning can be unstable, and changes such as different seeds can lead to large differences in task performance \parencite{mosbachStabilityFinetuningBERT2021}. It is therefore necessary to tune a set of key hyperparameters, which are settings that control how the model is trained. These are not learned by the model but must be set manually or through experimentation. Their values affect how fast the model learns, how stable training is, and how well the model generalizes to new data. The commonly tuned hyperparameters are briefly introduced below.

    The \textit{learning rate} controls how much the model updates its weights during each step \parencite{mosbachStabilityFinetuningBERT2021}. If it is too high, the model may not converge and instead jump over good solutions. If it is too low, training can be very slow or get stuck in local minima.

    \textit{Warmup steps} are used at the beginning of training to gradually increase the learning rate from zero to its target value \parencite{mosbachStabilityFinetuningBERT2021}. This helps avoid instability in the early stages, where large updates can be harmful. After the warmup period, the learning rate is often decreased again using a scheduler, which controls how it changes over time.

    The \textit{number of epochs} defines how many times the model passes through the entire training dataset \parencite{mosbachStabilityFinetuningBERT2021}. More epochs mean more training iterations, which can help the model better fit the data. On small datasets, training for too few epochs can cause underfitting, where the model does not learn enough from the data and performs poorly even on the training set. Training for more epochs, sometimes up to 20 instead of the usual 3, helps reduce underfitting and improves generalization. However, training for too many epochs can lead to overfitting, where the model learns the training data too closely and performs worse on new data.

    The \textit{batch size} refers to how many training examples the model processes before updating its parameters \parencite{mosbachStabilityFinetuningBERT2021}. Commonly, a batch size of 16 is used during fine-tuning \texttt{mBERT}. Larger batches provide more stable gradient estimates but require more memory. Smaller batches can introduce noise in the updates but might help the model generalize better. While \textcite{mosbachStabilityFinetuningBERT2021} does not deeply analyse batch size effects on stability, it remains an important parameter to balance resource limits and training quality.

    Finally, the \textit{optimizer} controls how the model weights are adjusted to minimize prediction error \parencite{mosbachStabilityFinetuningBERT2021}. The AdamW optimizer is standard for \texttt{mBERT} fine-tuning because it adapts learning rates per parameter and includes weight decay regularization. A critical feature of Adam is \textit{bias correction}, which reduces the effective learning rate early in training. This acts like an implicit warmup, preventing large unstable updates and vanishing gradients in the lower layers. Combining explicit warmup with Adam's bias correction allows training with higher learning rates more stably.

    \vspace{0.6em}
    \begin{table}[h]
        \centering
        \begin{tabularx}{\textwidth}{l X}
        \toprule
        \textbf{Hyperparameter} & \textbf{Role in Fine-Tuning} \\
        \midrule
        Learning Rate & Controls how much model weights are updated at each step; too high causes instability, too low slows training. \\
        Warmup Steps & Gradually increases the learning rate at the start to prevent unstable early updates. \\
        Number of Epochs & Defines how many times the model sees the full training data; more epochs help on small datasets. \\
        Batch Size & Number of samples processed before an update; affects stability, memory use, and generalization. \\
        Optimizer & Algorithm for updating weights; AdamW is standard, with adaptive rates and weight decay. \\
        \bottomrule
        \end{tabularx}
        \caption{Summary of key hyperparameters used during fine-tuning}
    \end{table}


\subsection{Layer Freezing} \label{subsection:layer_freezing}
    To speed up training and help prevent overfitting on small datasets, while preserving the broad language knowledge from pre-training, it is common to freeze certain layers of a pretrained model during fine-tuning. Layer freezing refers to keeping these layers fixed, meaning their weights are not updated. This reduces the number of trainable parameters \parencite{sorrentiSelectiveFreezingEfficient2023}. In monolingual BERT, lower layers typically encode general syntactic and semantic patterns, while higher layers are more task-specific \parencite{nadipalliLayerWiseEvolutionRepresentations2025}. As a result, lower layers are often frozen, and only the top layers and the classification head are fine-tuned, especially in resource-constrained settings \parencite{nadipalliLayerWiseEvolutionRepresentations2025}.


    In \texttt{mBERT}, the distribution of cross-lingual and language-specific features across all layers makes layer freezing less straightforward. \textcite{wuBetoBentzBecas2019} highlight that no single layer consistently captures the most relevant cross-lingual information, and even individual layers can perform well on sentence-level tasks. They suggest that freezing the lower six layers may improve generalization, but emphasize that optimal strategies depend on the specific task and require empirical testing \parencite{wuBetoBentzBecas2019}.

\subsection{Limitations of mBERT}
    One major limitation of \texttt{mBERT} is the "curse of multilinguality" \parencite{gurgurovMultilingualLargeLanguage2024}. Because it must represent 104 languages within a fixed parameter budget, the capacity available per language is limited. This causes reduced performance across languages compared to monolingual models. Even high-resource languages like English perform worse in \texttt{mBERT} than in their dedicated BERT models. Additionally, the shared vocabulary of 110,000 tokens is diluted, meaning it is less tailored to any single language. Languages with more data tend to get better performance, while others suffer. Since \texttt{mBERT} is pretrained on Wikipedia, it reflects biases inherent to that corpus. German Wikipedia articles predominantly use the generic masculine \parencite{sichlerGenderDifferencesGermanlanguage2014}, while gender-fair alternatives appear only sporadically, mostly in discussions or articles about female-dominated professions. These biases can influence the model's outputs and are especially important to consider in a gender bias detection context. Despite these limitations, \texttt{mBERT} remains the most fitting choice for this thesis. Since I work with English and German, which are both high-resource and related languages, \texttt{mBERT} generally performs better than it would with low-resource languages or languages from distant language families with fewer similarities \parencite{lauscherZeroHeroLimitations2020}.

\subsection{Evaluation Metrics}
    The fine-tuned BERT model must be evaluated to determine how accurately it detects gender bias. Evaluation metrics provide objective measures for assessing and comparing performance. In this task, it is especially important to reduce two types of errors: false positives, where unbiased translations are mistakenly flagged as biased, and false negatives, where genuine bias goes undetected. A model that guesses randomly or consistently avoids flagging bias offers little practical value. The metrics that capture these errors are precision and recall \parencite{rainioEvaluationMetricsStatistical2024}:

\begin{itemize}
    \item \textbf{Precision:} Of all translations flagged as biased, how many truly are biased? High precision means fewer false alarms.
    \item \textbf{Recall:} Of all biased translations, how many did the model correctly detect? High recall means fewer missed biases.
\end{itemize}

There is often a trade-off between precision and recall. A model with high precision but low recall misses many real biases, while one with high recall but low precision raises too many false warnings. To balance this trade-off, the F1 score is used. It combines precision and recall into a single number by calculating their harmonic mean:

\vspace{0.4em}
\[
F1 = 2 \cdot \frac{\text{Precision} \cdot \text{Recall}}{\text{Precision} + \text{Recall}}
\]
\vspace{0.4em}

Another common metric is accuracy, which measures the percentage of all translations that are classified correctly \parencite{rainioEvaluationMetricsStatistical2024}. Accuracy is straightforward and gives a sense of overall performance, but it can be misleading for imbalanced datasets. For example, if most translations are unbiased, a model that always predicts “unbiased” would achieve high accuracy but fail to identify any biased instances. The F1 score is better for evaluating the model and guiding selection because it focuses on the minority class of biased translations. Accuracy, however, remains useful as a complementary metric, particularly when assessing performance on a controlled test set. On the handcrafted test set, it shows how often the model predicts the correct label, giving a clear sense of its performance alongside the F1 score.
    \clearpage{\pagestyle{empty}\cleardoublepage}		
    \chapter{Related Work}

This section outlines key findings of related work on gender bias in MT, with a focus on the English-German language pair. The research aims are to (1) define the core concept of gender bias in MT, (2) establish the relevance of the topic, (3) identify the research gap, and (4) justify technical design choices. To support this, I examine datasets, model types, and tools used in previous studies.

For the literature review I combined incremental and conceptual literature review methods, where each source led to the identification of the next. Based on this progression, I identified key concepts and used them to organize and interpret the literature, aligning with a conceptual approach. The structure followed the qualitative Information Systems framework by \citet{schryenWritingQualitativeLiterature2015} and further informed by \citet{shresthaExploringGenderBiases2022}, who conducted a systematic review on gender bias in ML and AI. 

\section{Literature Search Process}

\subsection{Search Sources and Tools}
Sources were primarily searched on \href{https://scholar.google.com/}{Google Scholar} and \href{https://www.perplexity.ai/}{Perplexity}, which served as an additional search engine. Prompts and outputs from Perplexity have been saved and are included in the appendix. To organize and manage the collected sources, \href{https://www.zotero.org/}{Zotero} was used throughout the process.

\subsection{Literature Review Framing}

To answer the four research aims, I have defined the key concepts in \autoref{tab:key-concepts}. Key search terms consisted of \textit{gender bias}, \textit{machine translation}, \textit{AI}, \textit{machine learning}, \textit{German}, \textit{stereotypes}, and \textit{detection}. The focus was on literature published between 2019 and 2025 to maintain relevance and currency, while foundational and definitional works from earlier periods were selectively included. The initial search for the term \textit{gender bias in machine translation} returned over 18,000 results. Through my iterative selection process, this was narrowed down to 34 core sources.

\renewcommand{\arraystretch}{1.3}
\begin{table}[ht!]
\centering
\begin{tabularx}{\textwidth}{lX}
\toprule
\textbf{Key Concept} & \textbf{Description} \\
\midrule
Stereotypes and Biases in Society & Introduces the social foundations of bias by explaining how stereotypes form, persist, and shape expectations about gender roles. Necessary to understand why certain translation outputs may reflect or reinforce societal gender norms. \\

Machine and Algorithmic Bias & Explains how social biases can enter ML systems through data, design choices, or feedback mechanisms. Sets the groundwork for understanding how gender bias can emerge in translation models used in this thesis. \\

Gender Bias in English-German Translation & Focuses on the specific challenges of translating between English and German, where the lack of grammatical gender in English and its necessity in German can cause biased outputs. Defines the types of gender bias relevant to the classification task in this thesis. \\

Bias Detection Frameworks & Reviews existing methods for identifying gender bias in language data and ML outputs. Helps justify the choice of a classification approach for detecting bias in translations. \\

\bottomrule
\end{tabularx}
\caption{Key concepts relevant to this thesis}
\label{tab:key-concepts}
\end{table}

\subsection{Citation Tracking}
Backward citation searching involved reviewing references cited by selected papers, prioritizing frequently cited and foundational works relevant to gender bias in MT. Forward citation searching used Google Scholar's "cited by" function to identify newer research citing those key papers. Filtering with specific terms (e.g., \textit{German} and \textit{machine translation}) was applied during forward search to maintain focus. In addition to the main review process, supplementary sources were included as needed throughout the writing phase. These consist of contextual references, statistics, or secondary citations that support specific points but were not part of the core conceptual or methodological framework.

\subsection{Selection Criteria and Screening Process}
Titles and abstracts were manually screened to select relevant studies. Inclusion criteria required sources to specifically address gender bias in MT, provide examples or discussions of gender-related errors, or explain the significance of gender bias in this context. Exclusion criteria filtered out studies focusing on general NLP bias without a direct link to MT, non-gender biases without clear gender connection, and highly technical papers lacking contribution to the general understanding of gender bias. Full texts were reviewed after initial screening to confirm relevance and extract insights. Redundant sources not providing new perspectives aligned with the thesis goals were excluded.

\section{Empirical Evidence}
% say that it is well documented in MT
% Summarize how researchers measured bias (benchmarks, manual annotation, templates).


\section{Gender Bias in NLP}
% how it manifests
% how it affects it in this specific case
% refer to bias because of society
% how context affects it

\section{Other Approaches}
% compare how it has been approached with other languages
% what experiements they used and what systems
% GERMAN ENGLISH PAPERS
% - Overview of rule-based, ML-based, and LLM-based approaches.
    
\section{Limitations Identified by Prior Research}    


\section{Positioning of My Work}    
% - How your work builds on these findings.
% - What gap you address (binary classifier + demo).
    
%     **Use:** Comparison with prior methods.
    \clearpage{\pagestyle{empty}\cleardoublepage}		
    \chapter{Methodology}
The goal of this project is to develop a practical gender bias detection model tailored for real-world MT scenarios. It targets common use cases like translating everyday sentences or job descriptions, focusing on flagging biased language at the sentence level. This means the model evaluates each sentence independently, without considering context from surrounding sentences. This approach guides both the model’s design and the preparation of the training data, where each translation pair is treated as a separate example. The project begins by selecting and combining datasets from previous work (see \autoref{fig:workflow}). The model building phase then follows, as shown in the purple boxes. It starts with cleaning and preparing the data, followed by extracting features for training. A pre-trained \texttt{mBERT} model is then fine-tuned for the classification task. Its performance is measured using standard evaluation metrics. In the final step, the trained model is integrated  into the demo application.

\vspace{1cm} 
\begin{figure}[htb]
    \centering
    \scalebox{0.8}{\tikzstyle{startstop} = [rectangle, rounded corners, minimum width=3cm, minimum height=1cm,text centered, draw=black, fill=gray!20]
\tikzstyle{process} = [rectangle, minimum width=3cm, minimum height=1cm, text centered, draw=black, fill=blue!10]
\tikzstyle{arrow} = [thick,->,>=stealth]

\begin{tikzpicture}[node distance=1.7cm]

\node (start) [startstop] {Select and Combine Datasets};
\node (clean) [process, below of=start] {Clean and Pre-process Data};
\node (features) [process, below of=clean] {Initialize Model};
\node (train) [process, below of=features] {Train Model};
\node (evaluate) [process, below of=train] {Evaluate Model Performance};
\node (demo) [startstop, below of=evaluate] {Show Model in Demo Application};

\draw [arrow] (start) -- (clean);
\draw [arrow] (clean) -- (features);
\draw [arrow] (features) -- (train);
\draw [arrow] (train) -- (evaluate);
\draw [arrow] (evaluate) -- (demo);

\end{tikzpicture}
}
    \caption{Methodology Overview}
    \label{fig:workflow}
\end{figure}
\vspace{1cm} 

\section{Dataset}
    Since no ready-to-use dataset existed for this task and no prior work had developed a comparable model, it was necessary to define: (1) the required number of samples, and (2) the desired content of the dataset. For a binary classification task of detecting gender bias using \texttt{mBERT}, general guidelines suggest between 100 and 5,000 labeled samples for fine-tuning \parencite{pecherComparingSpecialisedSmall2024}, while multi-class tasks need fewer samples (around 100). However, the complex nature of gender bias often requires a larger dataset for robust detection since the number of samples depends mainly on the task type. For the final dataset, 2,000 to 5,000 samples were selected to provide enough data for effective training while staying within resource limits.

    Existing EN-DE datasets were reviewed to reduce the need for manual data creation. The following sources were considered: \texttt{\href{https://huggingface.co/datasets/FBK-MT/mGeNTE}{mGeNTE en-de}} \parencite{savoldiMGeNTEMultilingualResource2025}, \texttt{\href{https://github.com/g8a9/building-bridges-gender-fair-german-mt}{Building Bridges Dictionary}} \parencite{lardelliBuildingBridgesDataset2024}, and \texttt{\href{https://research.google/blog/a-dataset-for-studying-gender-bias-in-translation/}{Translated Wikipedia Biographies}} \parencite{stellaDatasetStudyingGender2021}.

    Analysis of the \texttt{Translated Wikipedia Biographies} dataset revealed several issues that prevented direct reuse. In many instances, the \texttt{perceivedGender} column contained subject names instead of expected labels such as \texttt{Male}, \texttt{Female}, or \texttt{Neutral}, making manual verification necessary. Additionally, all examples were labeled as neutral (0), as the dataset was designed around correctly gendered references. Since the remaining two datasets were already balanced and contained a sufficient number of neutrally gendered examples, the Wikipedia Biographies dataset was excluded from the final training data. \texttt{mGeNTE} contains naturally occurring sentences with gendered entities, while \texttt{Building Bridges} focuses on German GFL entries for explicitly gendered nouns such as professions. 

\begin{table}[ht!]
    \centering
    \renewcommand{\arraystretch}{1.3}
    \begin{tabularx}{\textwidth}{|>{\raggedright\arraybackslash}X|>{\raggedright\arraybackslash}X|>{\raggedright\arraybackslash}X|}
    \hline
    \textbf{Dataset} & \textbf{Description} & \textbf{Content} \\ \hline
    \texttt{mGeNTE en-de} \parencite{savoldiMGeNTEMultilingualResource2025} & Multilingual dataset to assess gender bias in MT. & \textasciitilde1,500 gender-ambiguous and gendered English sentences with gender-neutral and gendered German translations. \\ \hline
    \texttt{Building Bridges Dictionary} \parencite{lardelliBuildingBridgesDataset2024} & Bilingual dictionary designed to support gender-fair EN-DE translation. & \textasciitilde230 German gender-neutral and gender-inclusive singular and plural sentences with English equivalents. \\ \hline
    \end{tabularx}
    \caption{Overview of suitable EN-DE datasets based on past works}
    \label{tab:available_datasets}
\end{table}

\subsection{Pre-processing}

\subsubsection{mGeNTE en-de} 
The mGeNTE dataset contained the following relevant information:  

\begin{itemize}  
  \item \texttt{SET-G}: English sentences with a clearly gendered subject.  
  \item \texttt{SET-N}: English sentences with neutral or ambiguous subject gender.  
  \item \texttt{REF-G}: German translations that preserve or introduce gender.  
  \item \texttt{REF-N}: German translations that are fully gender–neutral.  
\end{itemize}  

\noindent
The bias definition used in this study classifies translations that omit the original gender as neutral, as they do not rely on a male default or stereotype. Although gender-neutral translations may be imperfect, they are not considered biased within this framework. Initial experiments indicated that including \texttt{REF-N} pairs during training led to over-penalization of neutral outputs. Due to the limited availability of neutral examples, such outputs were not penalized in the final training setup. Each original entry was split into two paired examples and labeled as follows:  
\[
\begin{aligned}
\texttt{SET\!-\!G} + \texttt{REF\!-\!G} &\;\rightarrow\; 0 \quad(\text{neutral})\\
\texttt{SET\!-\!G} + \texttt{REF\!-\!N} &\;\rightarrow\; 0 \quad(\text{neutral})\\
\texttt{SET\!-\!N} + \texttt{REF\!-\!N} &\;\rightarrow\; 0 \quad(\text{neutral})\\
\texttt{SET\!-\!N} + \texttt{REF\!-\!G} &\;\rightarrow\; 1 \quad(\text{biased})
\end{aligned}
\]  
This procedure yields 3,000 total instances, of which 750 are labeled biased (1) and 2,250 are labeled neutral (0).\footnote{The transformed dataset can be found in \texttt{/datasets/mgente\_final.csv}.} 

\subsubsection{Building Bridges Dictionary} 
This dataset consisted of a GFL dictionary of nouns, not full sentences. That made it useful for studying GFL, but not suitable for this task, which requires sentence-level context. To address this, prompt engineering was used with Google Gemini 2.5 Flash to synthetically expand the dataset\footnote{Refer to Appendix~\ref{appendix:gemini_prompt} for the prompt.}. Nouns from the original dataset were used to create multiple grammatically correct sentence variations, covering singular, plural, gender-neutral, and gender-inclusive forms. The dataset uses the star form (e.g., \textit{Lehrer*innen}) as its inclusive format. Since the colon form (e.g., \textit{Lehrer:innen}) is also common in practice, a script was used to duplicate all entries with stars and replace the star with a colon to generate additional variants. This resulted in 3,381 total entries: 2,001 labeled as 0 (neutral) and 1,380 labeled as 1 (biased).\footnote{The transformed dataset can be found in \texttt{/datasets/lardelli\_final.csv}.}

\subsubsection{Tatoeba} 

The aforementioned setup lacked genuinely neutral examples, defined as sentences without any gendered subject, such as "The weather is nice" or "How are you". Including such cases is important for training the model to recognise that not all translations are relevant for gender bias detection, and that many sentences should be classified as neutral. As no suitable dataset for this category was available, a supplementary set was created from random EN–DE sentence pairs drawn from the \texttt{\href{https://tatoeba.org/en/}{Tatoeba}} corpus. A total of 550 sentence pairs was sampled. Manual filtering was applied to these samples to remove any pairs containing incorrect translations or stereotypical gendering, as public contributions often default to male forms. The final subset contained 532 clearly neutral sentence pairs, all labelled with 0.\footnote{The transformed dataset can be found in \texttt{/datasets/tatoeba\_final.csv}.}


\subsubsection{Available Data Summary}

\autoref{tab:data-summary} shows an overview of the labeled data from the three available sources. 

\begin{table}[H]
\centering
\begin{tabularx}{\textwidth}{l *{3}{>{\centering\arraybackslash}X}}
\toprule
\textbf{Dataset} & \textbf{Total} & \textbf{Neutral (0)} & \textbf{Biased (1)} \\
\midrule
\texttt{lardelli\_final.csv} & 3381 & 2001 & 1380 \\
\texttt{mgente\_final.csv}   & 3000 & 2250 & 750  \\
\texttt{tatoeba\_final.csv}      & 532  & 532  & 0    \\
\bottomrule
\end{tabularx}
\caption{Summary of available labeled examples}
\label{tab:data-summary}
\end{table}

The number of samples selected from each dataset was determined through iterative testing. Multiple dataset variants were created by upsampling or downsampling specific groups. The documentation of this process is discussed in \autoref{subsection:hyperparameter_tuning_methodology}.

\subsection{Data Splitting and Cleaning}
    The dataset is partitioned into training (80\%), validation (10\%), and test (10\%) subsets. This splitting ratio follows established practices commonly employed in ML experiments \parencite{bahetiTrainTestValidation2021}. It provides enough samples for the model to learn general patterns while reserving separate subsets for tuning and final evaluation. Stratified sampling was used to maintain consistent label distribution (biased vs. neutral) across all three sets. For example, if 30\% of the full dataset is biased, each split will also have 30\% biased samples. 

    Advanced text cleaning steps (punctuation removal, lowercasing, or stemming) were not applied due to the use of \href{https://huggingface.co/google-bert/bert-base-multilingual-cased}{\texttt{bert-base-multilingual-cased}}. This tokenizer handles raw, unaltered text and retains case distinctions. The model was pretrained on large corpora containing natural language in its original form \parencite{devlinBERTPretrainingDeep2019}, so modifying the input by lowercasing or stripping punctuation could remove meaningful patterns the model has learned to recognize. Steps to handle missing values or invalid entries were already performed in the individual datasets, so they did not need to be repeated when creating the final merged dataset.

\section{Training Pipeline}
    \texttt{mBERT} with a binary classification head is used to predict whether a translation is \textit{biased} or \textit{neutral}. The tokenizer encodes input sentence pairs into numerical representations and distinguishes between source and target sentences. All sequences are padded or truncated to a fixed length of 256 tokens, which preserves most content while keeping processing efficient. The model represents each input pair with a summary vector of the entire sequence, which is then used for classification into the two categories. Each dataset is instantiated and encoded into a format suitable for model input, including the EN–DE sentence pairs and their corresponding labels. Training hyperparameters are established through tuning. The model iteratively learns from the training data by adjusting its parameters to minimize classification errors, with validation performance guiding the process and helping prevent overfitting. At the end of training, the best-performing model is selected based on validation metrics and used for subsequent gender bias detection.

\section{Evaluation Strategy}
    Model evaluation was performed during training using the validation set and after training using two distinct test sets. As detailed in \autoref{subsection:hyperparameters_explained}, the macro F1 score was employed as the primary metric to assess model performance. The validation set served to monitor training progress across epochs, and the checkpoint with the highest validation F1 score was saved. The combined training dataset was handcrafted and had known limitations, so relying solely on the validation set was insufficient to assess final model performance. To better evaluate generalization, a separate handcrafted test set was created. This set contains EN-DE sentence pairs with manually assigned bias labels. Using these two evaluation strategies, both the fine-tuning process and the composition of the combined training dataset were iteratively adjusted to improve model robustness and generalization.

\subsection{Handcrafted Test Set Construction} \label{subsection:eval_dataset}
    The handcrafted test set was developed from a user-centered perspective, focusing on identifying inputs that expose various failure and edge cases. Examples were organized into categories: neutral sentences, neutral sentences containing gendered roles, biased translations, and translations featuring German GFL. It comprises simple synthetic sentences written specifically for this purpose, as well as authentic examples extracted from job postings. The inclusion of real-world data aims to simulate practical use cases, such as evaluating translated job advertisements for gender bias.

    Emphasis was placed on diversity in sentence structure and content rather than maintaining label balance. Certain examples tested the model’s tendency to incorrectly flag neutral sentences containing gendered terms, while others assessed its capacity to detect various GFL forms in German, including terms like “Lehrende” and the colon notation “Lehrer:innen.” Bias labels were assigned manually according to the criteria established in Chapter 2. The complete handcrafted test set, containing 26 labeled translation pairs, is provided in the Appendix.

\subsection{Hyperparameter Selection and Tuning} \label{subsection:hyperparameter_tuning_methodology}
     While a few standard hyperparameters were tested, the focus was placed on tuning dataset composition and the number of frozen layers. These factors showed a significantly stronger influence on model performance during experimentation. Since the training data originated from a mix of external sources with varying quality, adjusting the use and structure of the data was considered more effective than extensive hyperparameter optimization. Recommended default values from prior work provided sufficiently strong baselines and were therefore used as the starting point.

    \paragraph{Epochs} The model was trained for a maximum of 8 epochs, with early stopping enabled using a patience of two epochs. This setup halted training if the macro F1 score did not improve over two consecutive epochs. The approach follows the recommendation by \textcite{pecherComparingSpecialisedSmall2024}, who suggest training until convergence, with a cap of 10 epochs and early stopping. In this case, validation loss typically increased after 8 epochs, with no further improvements observed. Limiting the training to 8 epochs helped mitigate overfitting and reduced training time.

    \paragraph{Batch size} A batch size of 16 was used. This value is commonly applied in fine-tuning scenarios involving small datasets, offering a reasonable balance between memory efficiency and training stability. Smaller batch sizes paired with lower learning rates were tested but led to reduced performance and less effective learning in early epochs. Existing literature, including \textcite{mosbachStabilityFinetuningBERT2021}, supports the use of a batch size of 16; no further experiments with smaller values were conducted.

    \paragraph{Learning rate} The learning rate was set to 2e-5. This value, originally proposed in \textcite{devlinBERTPretrainingDeep2019}, remains widely used for fine-tuning transformer models. Alternatives such as 1e-5 and 3e-5 were evaluated but yielded slightly lower validation scores. The 2e-5 setting showed the most stable and consistent results and was therefore applied in all final training runs.

    \paragraph{Optimizer and scheduler} The \texttt{Trainer} API employed the AdamW optimizer by default. A warmup-linear learning rate schedule was used: the rate gradually increased during the first 10\% of training steps (warmup) and then decreased linearly until completion. This schedule supports smooth learning and helps prevent instability during early training.

\subsection{Training Dataset Tuning} \label{subsection:training_dataset_tuning}
    Since the F1 scores were similar across dataset versions, the main evaluation was based on the handcrafted test set of 26 sentences.\footnote{The detailed documentation of each iteration is included in Appendix \ref{appendix:dataset_tuning_table}. The focus in this section is on the process and rationale rather than on numeric results.} This small test set does not provide a complete indication of overall model quality, but it offers insight into practical usability. Any statements regarding better or worse performance should be considered in light of this limitation.

    The dataset \texttt{mgente\_final} was considered the best source because its samples are natural sentences. All 750 biased samples from \texttt{mgente\_final} were included, along with exactly 750 neutral samples. The tuning process aimed to adjust the remaining datasets to maintain a maximum ratio of 60\% neutral to 40\% biased samples. Sampling was performed using the \texttt{join\_datasets.py} script, which loads the labeled datasets, samples a fixed number of biased and neutral entries per dataset using a fixed random seed (10), and combines them into a single training set. The script also checks for missing values and label integrity before saving the final CSV file. The tuning process across dataset versions, including their composition and rationale, is summarized in \autoref{tab:dataset_versions}.

    The Baseline dataset already achieved strong performance, with a test F1 score of 0.975 and 84.6\% accuracy on the handcrafted test set. However, it failed on some neutral examples such as \textit{"My mother is an engineer." / "Meine Mutter ist Ingenieurin."} (predicted biased with confidence 0.55) and on certain German GFL patterns (e.g., double naming and the colon notation). Adjustments made to the dataset composition in Datasets B through E occasionally improved specific weaknesses. In one case, Dataset E succeeded in correctly classifying a neutral gendered sentence that the Baseline had misclassified. At the same time, these targeted improvements often introduced new issues, such as misclassifications in job advertisement examples. The changes did not produce consistent gains on the handcrafted test set and in some cases reduced overall accuracy. As a result, the Baseline dataset composition was used for final training, as it offered the most reliable balance between targeted performance and general usability.

\vspace{0.8em}
\begin{table}[ht]
    \centering
        \begin{tabularx}{\textwidth}{l X >{\raggedright\arraybackslash}X}
    \toprule
    \textbf{Dataset} & \textbf{Rationale} & \textbf{Sample Distribution (biased/neutral)} \\
    \midrule
    A & Initial setup using equal parts of \texttt{mgente\_final} and \texttt{lardelli\_final}, with some \texttt{tatoeba\_final} neutrals. & mgente 750 / 750, lardelli 750 / 750, tatoeba 0 / 250 \\
    B & Built on A. Increased \texttt{lardelli\_final} neutrals to better capture GFL patterns and added more \texttt{tatoeba\_final} neutrals. & mgente 750 / 750, lardelli 750 / 1000, tatoeba 0 / 400 \\
    C & Built on A and B. Reduced \texttt{lardelli\_final} biased examples to counter possible overrepresentation. & mgente 750 / 750, lardelli 400 / 750, tatoeba 0 / 250 \\
    D & Built on A and C. Further improved neutral recognition by adding more \texttt{tatoeba\_final} neutral sentences. & mgente 750 / 750, lardelli 750 / 750, tatoeba 0 / 500 \\
    E & Built on A and C. Increased \texttt{mgente\_final} neutral data to raise diversity from naturalistic examples. & mgente 750 / 1,250, lardelli 750 / 750, tatoeba 0 / 250 \\
    \bottomrule
    \end{tabularx}
    \caption[Dataset iterations with rationale and composition]{Dataset iterations with rationale and composition. Each version builds on the Baseline and previous adjustments. Format: source biased / neutral}
    \label{tab:dataset_versions}
\end{table}


\subsection{Layer Freezing Tuning}
    All dataset tuning experiments described above were conducted with layer freezing set to $n=8$, meaning that encoder layers 0 through 7 of \texttt{mBERT} were frozen during training. As explained in \autoref{subsection:hyperparameters_explained}, earlier studies have shown that the middle layers are most semantically informative, while lower layers tend to capture syntactic information. Freezing up to layer 8 was chosen as a baseline to reduce training time while still allowing the model to adapt higher-level representations to the task.Since the results with $n=8$ were already promising, only two further variations were tested: $n=7$ and $n=6$. These settings freeze fewer layers, meaning more of the network remains trainable. The purpose of these tests was to evaluate whether this added flexibility improved performance without overfitting.

    \vspace{0.8em}
    \begin{table}[ht]
        \centering
        \begin{tabularx}{\linewidth}{Xcc}
        \toprule
        \textbf{Frozen Layers} & \textbf{Test F1 (weighted)} & \textbf{Handcrafted Test Set Accuracy} \\
        \midrule
        $n=6$ (layers 0--5 frozen) & 0.981 & 0.808 \\
        $n=7$ (layers 0--6 frozen) & 0.979 & 0.808 \\
        $n=8$ (layers 0--7 frozen) & 0.966 & 0.846 \\
        \bottomrule
        \end{tabularx}
        \caption{Comparison of layer freezing settings}
    \end{table}

    Freezing fewer layers led to slightly higher F1 scores on the test set, but the model with $n=8$ frozen layers achieved the best results on the handcrafted test sentences, which were designed to reflect real-world usability. Since the F1 differences were minor and freezing more layers results in a simpler and more efficient model, $n=8$ was chosen as the final setting.

\section{Demo Application Design}
    The demo application comprises three modules: the user interface, the bias detection model, and the translation component (see \autoref{fig:component_diagram}). Both input modes converge on a common prediction pipeline. On launch, the application loads the trained model and its encoding mechanism into memory. Users can either input raw English text (Tab 1), which is split into sentences and translated into German, or provide aligned EN-DE sentence pairs (Tab 2). Both approaches result in a set of sentence pairs ready for bias analysis. The bias detection pipeline processes these sentence pairs by converting each pair into a representation suitable for the model, which in turn predicts whether a translation is biased or neutral. Predictions are accompanied by a confidence score, which is the probability assigned by mBERT to the predicted label. Results are displayed alongside the original and translated sentences. If bias is detected above a defined threshold, a warning is shown; otherwise, the sentence pair is marked as neutral. Each result is separated to maintain readability.

    \vspace{0.8em}
    \begin{figure}[htb]
    \centering
    \includegraphics[width=1\textwidth]{modules.png}
    \caption[Component Diagram]{Component Diagram. High-level architecture of the bias detection demo app showing components, their relationships, and data flow}
    \label{fig:component_diagram}
    \end{figure}

    The interface has two main modes (see Figures~\ref{fig:sequence_diagram_1}, \ref{fig:sequence_diagram_2}). In automatic translation, raw English text is submitted, segmented into sentences, and translated. Each sentence is paired with its translation before analysis. In manual pairing, users provide EN-DE sentence pairs directly, which bypasses translation and proceeds to bias evaluation. The system is designed to demonstrate the end-to-end process of bias detection independently of implementation details. The methodology allows the translation component or model representation to be replaced with alternative approaches. The focus is on showing how input sentences are converted into pairs, analyzed for bias, and presented in a user-friendly way. 

    \vspace{0.8em}
    \begin{figure}[H]
    \centering
    \includegraphics[width=0.95\textwidth]{modules_001.png}
    \caption[Sequence Diagram – Tab 1 (with translation)]{Sequence Diagram – Tab 1 (with translation). Step-by-step flow for Tab 1, showing how user input is translated, tokenized, and processed by the bias detection model}
    \label{fig:sequence_diagram_1}
    \end{figure}

    \begin{figure}[H]
    \centering
    \includegraphics[width=0.95\textwidth]{modules_002.png}
    \caption[Sequence Diagram – Tab 2 (manual input)]{Sequence Diagram – Tab 2 (manual input). Step-by-step flow for Tab 2, showing how manually entered English and German sentences are tokenized and processed by the bias detection model}
    \label{fig:sequence_diagram_2}
    \end{figure}

    \clearpage{\pagestyle{empty}\cleardoublepage}		
    \chapter{Implementation}
Based on the methodology, this chapter presents the specific implementation carried out in this project, along with step-by-step instructions to run the demo and reproduce the results. The complete project repository, including all datasets and scripts, is available at \url{https://github.com/phmkhali/bias-detector-en-de}.

\section{Environment Setup and Project Structure}
    \subsection{System Environment and Hardware}
    The project was developed on macOS using Python 3.12.4. A virtual environment was used, and all dependencies were installed via the \texttt{requirements.txt} file using \texttt{pip3}. No manual installation steps were needed beyond this. During development, both GPU and CPU were used to test training and inference performance, with device selection handled dynamically. The final model for the demo and evaluation was trained on the CPU to ensure full compatibility and reproducibility without relying on a GPU. To ensure reproducibility, random seeds were fixed across all libraries and backends. The application is started via Streamlit. Further usage instructions are provided in \autoref{section:reproduction_guide}.

\subsection{Directory Layout}
    \autoref{fig:file_tree} shows the directory structure with the relevant files for the final implementation. The folder contains additional files related to the original datasets and scripts used for data conversion. These supplementary files are intended for comprehension and reproducibility purposes but are not required for the final model.

    \vspace{0.8em} 
    \begin{figure}[htb]
        \centering
        \scalebox{0.8}{\begin{forest}
  for tree={
    font=\ttfamily,
    grow'=0,
    child anchor=west,
    parent anchor=south,
    anchor=west,
    calign=first,
    edge path={
      \noexpand\path [draw, \forestoption{edge}]
      (!u.south west) +(7.5pt,0) |- (.child anchor) \forestoption{edge label};
    },
    before typesetting nodes={
      if n=1
        {insert before={[,phantom]}}
        {}
    },
    fit=band,
    before computing xy={l=15pt},
  }
[bias-detector-en-de
    [datasets
      [dataset.csv]
      [join\_datasets.py]
      [lardelli\_final.csv]
      [mgente\_final.csv]
      [tatoeba\_final.csv]
    ]
    [model\_output]
    [app.py]
    [fine-tuning.ipynb]
    [translate.py]
    [utils.py]
  ]
\end{forest}

}
        \caption[Relevant files of the final implementation]{Relevant files of the final implementation}
        \label{fig:file_tree}
    \end{figure}
    \vspace{0.8em} 


\section{Core Components and Data Flow}
    \subsection{Datasets Folder}
        The \texttt{datasets} folder contains the processed datasets used for training and evaluation of the bias detection model. It includes the final combined dataset \texttt{dataset.csv} as well as intermediate datasets derived from individual sources: \texttt{lardelli\_final}, \texttt{mgente\_final}, and \texttt{tatoeba\_final}. The \texttt{join\_datasets} script facilitates the concatenation of these datasets into a unified dataset. The script is designed to support iterative sampling by specifying sample sizes for biased and neutral classes with a fixed random seed. It merges the sampled subsets, shuffles the combined data, and performs data integrity checks. 

  
    \subsection{Fine-tuning Notebook}
        The \texttt{fine-tuning.ipynb} notebook carries out the complete process of preparing and fine-tuning the bias detection model. The individual steps have already been described in detail throughout this thesis. This section therefore serves as a summary of the implementation decisions in one place to provide a clearer picture of the overall workflow, without repeating the technical explanations of the underlying functionalities.  

        The process begins with loading the dataset and determining the computational device, prioritizing GPU resources if available. Device information is printed to confirm the configuration. The tokenizer corresponding to the selected mBERT model (\texttt{bert-base-multilingual-cased}) is then loaded. This tokenizer converts the EN-DE input sentences into token IDs required by the model. The model itself is instantiated with a classification head for two labels (\texttt{neutral} and \texttt{biased}) and subsequently moved to the selected device. The dataset is wrapped in a custom \texttt{BiasDataset} class. This class handles the conversion of EN-DE sentence pairs and their bias labels into tensor representations. Each item in the dataset is tokenized with truncation and padding up to a maximum length of 256 tokens. After defining this structure, the data is split into training and validation sets, and corresponding \texttt{BiasDataset} objects are created.  

        The training setup is then configured. Hyperparameters include a learning rate of \texttt{2e-5}, a batch size of 16, and a maximum of eight epochs. The training configuration is defined via \texttt{TrainingArguments}, which specify evaluation and logging strategies at epoch intervals, checkpoint saving, and early stopping. The best-performing model is selected based on the macro-averaged F1 score. Evaluation metrics are computed with a dedicated function that returns precision, recall, and F1 scores per class, along with macro-averaged values and accuracy. Finally, the training is executed using the \texttt{Trainer} class. The model, datasets, training arguments, and metric function are passed together, with an early stopping callback to prevent overfitting. The training process feeds batches of data to the model, updates weights, and evaluates the results after each epoch. Early stopping limited the training to six epochs, with a runtime of approximately 18 minutes and 32 seconds on the development machine. The final outputs, including checkpoints, configuration files, tokenizer, and training metadata, are stored in the \texttt{model\_output} folder, with the best-performing model retained for subsequent use in bias detection.

        \subsection{Evaluation on Test Sets}
    After training, the model is evaluated on the two test sets. The evaluation on the held-out set is performed using a dedicated function that computes standard classification metrics. A function is first defined to calculate the confusion matrix and to derive false positive and false negative rates. The \texttt{evaluate\_and\_print} function integrates this computation with Hugging Face's \texttt{Trainer} interface. It prints the macro-averaged F1 score, the confusion matrix breakdown, the error rates, and a detailed classification report for the given dataset. To further examine model weaknesses, a detailed error analysis is carried out. Predictions are stored together with the corresponding EN-DE sentence pairs and their true labels. From this, false positives and false negatives are extracted and presented in tabular form. 

    The handcrafted sentences are wrapped in the \texttt{BiasDataset} class and passed through the model one by one. For each sentence pair, the model outputs the predicted label, the true label, and the class probabilities. These results are collected in a table that shows the English text, the German translation, the ground truth, the predicted outcome, and whether the prediction was correct.  

    \subsection{Streamlit Application}
        The \texttt{app.py} file is the main entry point for the Streamlit interface.\footnote{Parts of the code were generated with AI and subsequently refined by the author; see Appendix \ref{appendix:ai_code} for details.}
        It loads the fine-tuned model and tokenizer from the \texttt{model\_output} directory and sets up the classification pipeline. In the automatic tab, users enter English text, which is translated using \texttt{translate.py}. In the manual tab, users enter both the original and translated sentences directly.\footnote{Screenshots of these tabs are shown in Figures~\ref{fig:demo_tab_1}, \ref{fig:demo_tab_2}, and \ref{fig:demo_multi_sentence}.} \texttt{translate.py} wraps the pre-trained OpusMT EN-DE model from Hugging Face. The main function, \texttt{translate\_batch}, takes a list of English sentences, tokenizes them, and uses the model's \texttt{generate} method to return a list of decoded German translations. The resulting sentence pairs from both tabs are passed to functions in \texttt{utils.py}. This module handles sentence splitting, optional translation, and model inference. It processes each sentence pair, predicts whether it contains gender bias, and formats the output with confidence scores for display in the interface.

        \vspace{0.8em}
        \begin{figure}[H]
            \centering
            \includegraphics[width=0.75\textwidth]{demo_tab_1.png}
            \caption[Streamlit Demo: Automatic Translation Tab]{Streamlit Demo: Automatic Translation Tab showing correct identification of bias in stereotypical occupational gender assignment}
            \label{fig:demo_tab_1}
        \end{figure}
        \vspace{0.8em}

        \begin{figure}[H]
            \centering
            \includegraphics[width=0.75\textwidth]{demo_tab_2.png}
            \caption[Streamlit Demo: Manual Translation Tab]{Streamlit Demo: Manual Translation Tab showing no bias detected due to use of GFL}
            \label{fig:demo_tab_2}
        \end{figure}
        \vspace{0.8em}

        \begin{figure}[H]
            \centering
            \includegraphics[width=0.8\textwidth]{demo_multi_sentence.png}
            \caption[Streamlit Demo: Multi Sentence Translation]{Streamlit Demo correctly splitting and labeling two sentences as neutral and biased}
            \label{fig:demo_multi_sentence}
        \end{figure}
        \vspace{0.8em}

\section{Reproduction Guide} \label{section:reproduction_guide}
   The setup process for the Streamlit demo app includes creating a Python virtual environment, installing required packages, and running the application. This guide covers these steps for macOS/Linux and Windows. \textbf{Note:} The pre-trained model is not included in the GitHub repository due to size restrictions. It can be downloaded separately via the provided Google Drive link.\footnote{ \href{https://drive.google.com/drive/folders/11WMb0od_U_sQsUGD0t4DjQwcefI3r_kK?usp=sharing}{Google Drive link} for the model.} If the pre-trained model is not present, the \texttt{fine-tuning.ipynb} notebook must be executed first to train and save the model before launching the demo.
\paragraph{Installation Steps}

\begin{enumerate}
    \item Open a terminal (macOS/Linux) or PowerShell (Windows).
    
    \item Clone the GitHub repository and download the pre-trained model:
        
    \begin{tcolorbox}[colback=gray!10, colframe=gray!50, breakable, boxrule=0.4pt, sharp corners]
\begin{verbatim}
git clone https://github.com/phmkhali/bias-detector-en-de
cd bias-detector-en-de
\end{verbatim}
    \end{tcolorbox}
    
    \textit{Download the model}
    
    \begin{tcolorbox}[colback=gray!10, colframe=gray!50, breakable, boxrule=0.4pt, sharp corners]
\begin{verbatim}
# Manually download from the provided Google Drive link above
# and place the model_output folder into the directory
\end{verbatim}
    \end{tcolorbox}
    
    \item Create and activate a Python virtual environment:
    
    \textit{macOS / Linux}
    
    \begin{tcolorbox}[colback=gray!10, colframe=gray!50, breakable, boxrule=0.4pt, sharp corners]
\begin{verbatim}
python3 -m venv venv
source venv/bin/activate
\end{verbatim}
    \end{tcolorbox}
    
    \textit{Windows}
    
    \begin{tcolorbox}[colback=gray!10, colframe=gray!50, breakable, boxrule=0.4pt, sharp corners]
\begin{verbatim}
python -m venv venv
.\venv\Scripts\activate
\end{verbatim}
    \end{tcolorbox}
    
    \item Install the required packages:
    
    \textit{macOS / Linux}
    
    \begin{tcolorbox}[colback=gray!10, colframe=gray!50, breakable, boxrule=0.4pt, sharp corners]
\begin{verbatim}
pip3 install -r requirements.txt
\end{verbatim}
    \end{tcolorbox}
    
    \textit{Windows}
    
    \begin{tcolorbox}[colback=gray!10, colframe=gray!50, breakable, boxrule=0.4pt, sharp corners]
\begin{verbatim}
pip install -r requirements.txt
\end{verbatim}
    \end{tcolorbox}
    
    \item (Only necessary if the \texttt{model\_output} directory was not downloaded and added to the repository) Run the \texttt{fine-tuning.ipynb} notebook manually to generate the model.

    \item Run the Streamlit app:
    
    \textit{macOS / Linux}
    
    \begin{tcolorbox}[colback=gray!10, colframe=gray!50, breakable, boxrule=0.4pt, sharp corners]
\begin{verbatim}
python3 -m streamlit run app.py
\end{verbatim}
    \end{tcolorbox}
    
    \textit{Windows}
    
    \begin{tcolorbox}[colback=gray!10, colframe=gray!50, breakable, boxrule=0.4pt, sharp corners]
\begin{verbatim}
streamlit run app.py
\end{verbatim}
    \end{tcolorbox}
\end{enumerate}

    \clearpage{\pagestyle{empty}\cleardoublepage}		
    \chapter{Evaluation and Findings}
    This chapter focuses on the interpretation of performance metrics and typical error patterns by analysing typical error patterns in the test sets and conducting exploratory tests to validate initial assumptions.

\section{Model Performance}
    The model’s overall accuracy on the held-out test set is 0.966, with matching macro F1 and weighted average. This shows it performs evenly across both classes. Precision and recall reveal biased cases are detected very well (recall 0.993) with some false alarms (precision 0.937), meaning a small portion of neutral cases are incorrectly labeled as biased. The confusion matrix confirms this with 10 false positives and only 1 false negative in 325 examples. The false positive rate is 0.057 and the false negative rate is 0.007. Overall, this suggests the model favors detecting bias at the risk of some over-flagging, which fits typical needs for bias detection where missing bias is costlier than false alarms.

        \vspace{0.8em}
        \begin{table}[H]
            \centering
            \begin{tabular}{lccc}
            \toprule
            \textbf{Class} & \textbf{Precision} & \textbf{Recall} & \textbf{F1 Score} \\
            \midrule
            Neutral (0) & 0.994 & 0.943 & 0.968 \\
            Biased (1)  & 0.937 & 0.993 & 0.964 \\
            \bottomrule
            \end{tabular}
            \caption{Per-class precision, recall, and F1 score on the test set}
        \end{table}

        \vspace{0.8em}
        \begin{table}[H]
            \centering
            \begin{tabular}{lc}
            \toprule
            \textbf{Metric} & \textbf{Value} \\
            \midrule
            Accuracy & 0.966 \\
            Macro-average F1 Score & 0.966 \\
            Weighted-average F1 Score & 0.966 \\
            False Positive Rate & 0.057 \\
            False Negative Rate & 0.007 \\
            \bottomrule
            \end{tabular}
            \caption{Overall evaluation metrics on the test set}
        \end{table}

        \begin{figure}[ht]
            \centering
            \includegraphics[width=0.75\textwidth]{test_set_confusion_matrix.png}
            \caption[Confusion matrix on the test dataset]{Confusion matrix of the model on the test dataset, showing true vs. predicted labels with counts}
            \label{fig:test_confusion_matrix}
        \end{figure}

    False positives and false negatives make up only a small share of the total predictions, but they point to recurring patterns in the model’s errors. These cases were grouped by shared features to enable a structured analysis of likely causes.\footnote{Refer to the appendix for all misclassified examples, including English source texts and German translations.}

    \paragraph{State and Governance Entity Terms}

    Sentences containing political, legal, or governance-related terminology were misclassified the most (6/10 cases). Terms such as president, police officers, heads of state and government, and political leaders were flagged as biased, despite the absence of gendered language in the German translation. The model may struggle to distinguish between genuinely biased constructions and content related to institutional or geopolitical domains. Another possible explanation is the strong male association of such terms in real-world data, which may influence model behavior through learned co-occurrence patterns \parencite{kroeberItsLongWay2022}.

    \paragraph{Training Dataset Error Causing a False Positive}

    One false positive appears to stem from an inconsistency within the training dataset. The English sentence "[...] or would you go to a surgeon?" was translated as "[...] oder von einem Menschen, der als Chirurg ausgebildet wurde [...]?". This example originates from the mGeNTE dataset and is labeled as unbiased. The German translation attempts to avoid using "der Chirurg" (m.) by phrasing it as "a person trained as a surgeon," presumably to circumvent the generic masculine. However, the word "Chirurg" still appears in the translation in its masculine form. The model correctly identified this gendered term and flagged it, despite the label indicating neutrality. From a bilingual perspective, the model’s detection is a true positive and the false flagging points to a labeling inconsistency in the dataset.

    \paragraph{Religious identities}
    
    One false negative involved a sentence concerning the immigration of muslims. "Muslims" was translated using the generic masculine form "Muslime", yet the model failed to recognize this instance as biased. Since the sentence does not provide a clear reason for why it was misclassified, this issue will be further analysed using exploratory testing (\autoref{section:exploratory_testing}).

    \paragraph{Issues with GFL and Semantically Gendered Terms}

    Two instances were incorrectly flagged as biased due to the use of GFL. The gender-ambiguous terms "specialists" and "recipient" were translated as "Sachkundigen" and "Rezipierende," respectively, both using neutral rewording. The false detection here reflects insufficient understanding of GFL forms.

    Additionally, one semantically gendered term, "uncle," translated as "Onkel," was flagged, pointing towards difficulty in distinguishing inherently gendered terms from translation-induced gender bias. This assumption will be tested for consistency in \autoref{subsection:generalization_performance_on_unseen}.

\section{Generalization performance on unseen data} \label{subsection:generalization_performance_on_unseen}
    To assess how well these findings generalize, the results from the handcrafted test set are now analysed. As explained in \autoref{subsection:eval_dataset}, the handcrafted test set was curated to expose potential failure modes and edge cases that may not be captured by standard evaluation.\footnote{Refer to the appendix for the handcrafted test results and confidence scores.}

    \subsection{Weaknesses}
    \autoref{tab:false_positives_with_conf} shows all of its false predictions. Consistent with the results from the held-out test set, these errors involve semantic gender distinctions and GFL variants. It is important to note that the test set did include more semantic gender examples, such as “sister,” “boy,” and “father” in ambiguous contexts, which the model did not flag. Sentence 5 with “mother” was flagged, but with low confidence. It can be concluded that this limitation stems from insufficient training data for semantically gendered words. The model recognizes a pattern but lacks enough instances to confidently classify. The remaining three cases share the same sentence structure but use different GFL forms. The model flagged some sentences using the pair form and gender-inclusive characters (the colon) as biased, with over 95\% confidence. It did however correctly identify the user of the gender star and neutral rewording of "teacher" to "Lehrende" as unbiased, also with a confidence of over 95\%.
    
    This reflects the known limitation of the dataset, where the limited availability of high-quality natural GFL data required supplementing it with lower-quality synthetic examples. Still, the correct classification in some cases suggests that the model may be capable of handling GFL more reliably if more representative data were available.

    \vspace{0.8em}
        \begin{table}[ht]
        \centering
        \begin{tabular}{|c|p{5cm}|p{5cm}|c|c|}
        \hline
        \textbf{Index} & \textbf{English} & \textbf{German} & \makecell[c]{Neutral \\ Prob \%} & \makecell[c]{Biased \\ Prob \%} \\
        \hline
        5 & My mother is an engineer. & Meine Mutter ist Ingenieurin. & 44.1 & 55.9 \\
        16 & Engineers are kind and hard-working & Ingenieure und Ingenieurinnen sind lieb und fleißig & 0.08 & 99.92 \\
        18 & Teachers are kind and hard-working & Lehrer:innen sind lieb und fleißig & 3.1 & 96.9 \\
        19 & Teachers are kind and hard-working & Lehrerinnen und Lehrer sind lieb und fleißig & 0.05 & 99.95 \\
        \hline
        \end{tabular}
        \caption{Handcrafted test sentences with incorrect model predictions and confidence scores (percent)}
        \label{tab:false_positives_with_conf}
        \end{table}

\subsection{Strengths}
The performance metrics earlier already imply that the model performs strongly in detecting genuinely biased cases. This is supported by the handcrafted test set: no clearly biased sentence was missed, and out of these, not a single one received a confidence score below 90\%. The model also demonstrates the ability to distinguish between clearly gendered and neutral content. Generic sentences such as “How are you?”, translated as “Wie geht es dir?”, were correctly classified as neutral. Likewise, sentences taken from real job postings were consistently identified with high confidence (e.g., “The ideal candidate” → “Der ideale Kandidat” (m.) = biased). In total, all tested job posting examples reached confidence scores above 99\%. This indicates that the model is well-suited for handling practical use cases, which supports its applicability in real-world scenarios.

\section{Exploratory Testing} \label{section:exploratory_testing}
    To further investigate the unresolved misclassification of the religious identity sentence, the demo was used to test additional cases. The term "Muslims" was replaced with other religious terms, followed by "soldiers" to represent a non-religious but contextually relevant group, and concluded with "doctors," a typical example of a generic masculine translation.

    \vspace{0.8em}
    Sentence Under Investigation:
    \begin{quote}
    \textbf{English:} Here too the local people are frustrated by the immigration of Muslims and the hard line taken by the military.

    \textbf{German:} Hier wird die lokale Bevölkerung ebenfalls durch die Zuwanderung von Muslimen und das unnachsichtige Auftreten des Militärs schwer gebeutelt.
    \end{quote}

        \vspace{0.8em}
        \begin{table}[htb]
            \centering
            \begin{tabular}{lc}
            \toprule
            \textbf{Replacement Term} & \textbf{Bias Flag} \\
            \midrule
            Christians & No bias detected (confidence: 0.89) \\
            Jews & No bias detected (confidence: 0.93) \\
            Soldiers & No bias detected (confidence: 0.79) \\
            Doctors & No bias detected (confidence: 0.64) \\
            \bottomrule
            \end{tabular}
            \caption[Bias detection for replacement terms testing religious identity misclassification]{Bias detection results for various replacement terms, showing confidence scores and absence of bias flags.}
        \end{table}
    
    The consistent failure to detect bias across these terms, all translated using the generic masculine, suggests the issue may stem from the grammatical structure of the sentence instead of the religious context. 

    Testing formal features like punctuation and casing showed a clearer impact on the confidence. Removing the period dropped confidence to 0.84. Lowercasing "Muslims" lowered it to 0.83. Doing both together caused the confidence to fall to 0.58. Because the model is based on cased BERT, it relies on correct punctuation and capitalization for context. The limited training data increases the model’s sensitivity to such formal changes, causing performance to vary when cues like casing and punctuation are missing or altered. To confirm this effect, the insertion tests were repeated with both punctuation and casing removed. In these modified cases, "christians" was flagged as biased (confidence: 0.98), "soldiers" was not flagged (confidence: 0.63), and "doctors" was again flagged (confidence: 0.76).

    \vspace{0.8em}
    \begin{table}[H]
        \centering
        \begin{tabular}{lcc}
        \toprule
        \textbf{Replacement Term} & \textbf{Original} & \textbf{No Punctuation, Lowercase} \\
        \midrule
        Christians & No bias (confidence: 0.89) & Bias (confidence: 0.98) \\
        Jews & No bias (confidence: 0.93) & No bias (confidence: 0.51) \\
        Soldiers & No bias (confidence: 0.79) & No bias (confidence: 0.63) \\
        Doctors & No bias (confidence: 0.64) & Bias (confidence: 0.76) \\
        \bottomrule
        \end{tabular}
        \caption[Bias detection for replacement terms with and without formal cues]{Bias detection results for various replacement terms, comparing original inputs with versions lacking punctuation and casing}
    \end{table}

    One final hypothesis was that the term "local people" may influence the classification. In the original translation, it was rendered as "lokale Bevölkerung"; a neutral term that may give the impression of a gender-neutral subject. This could have caused the model to focus less on the gendered translation of "Muslims". To explore this, the term "local" was removed from the original sentence, changing the translation to "Auch hier sind die Menschen durch die Einwanderung von Muslimen und die harte Linie des Militärs frustriert."\footnote{This translation slightly differs in wording because it was generated using OpusMT. The version in the test set was likely manually translated by the mGeNTE creators.} The model correctly flagged this version as biased (confidence: 0.97). The presence of both a neutral and a gendered subject within the same sentence should be considered a factor that may reduce the model’s classification accuracy. At the same time, testing edge cases is inherently challenging due to the interaction of multiple subtle influences. While stepwise sentence modifications can help identify possible sources of misclassification, definitive conclusions remain difficult to establish.

    In this chapter, the model was evaluated in view of the research question. Its performance was analyzed on both a held-out dataset and a handcrafted dataset of unseen examples. Real-world texts, such as job postings, were used to test how the model performs in practice. Common error patterns were identified and grouped to better understand factors that influence misclassification, providing a basis for the conclusions in the next chapter.


 
    \clearpage{\pagestyle{empty}\cleardoublepage}		
    \chapter{Conclusion and Discussion}
   Without a doubt, MT systems and their accessibility drastically improve our ability to communicate with one another. While modern NMT provides semantically accurate translations for high-resource languages, they often introduce gender bias when translating between languages with and without grammatical gender. This thesis attempted to create an application to detect gender bias in EN-DE translations in real time. The findings show that an mBERT model fine-tuned for this binary classification task can detect gender bias with an accuracy of 96.6\% on a held-out test set, which comes from a train–test split of a combined dataset built from existing studies. On a separate handcrafted dataset designed to test edge cases, the model reaches 84.6\% accuracy. The model shows strong performance in identifying generic masculines and the assignment of stereotypical roles, which make up the majority of gender biases found in MT systems \parencite{lardelliBuildingBridgesDataset2024,stanovskyEvaluatingGenderBias2019,pratesAssessingGenderBias2019}. Despite its high accuracy, there are recurring error patterns that affect the model’s reliability. The error analysis revealed four sources of occasional incorrect predictions: (1) misclassifying German GFL forms as biased, (2) failing to detect semantically gendered words as unbiased, (3) failing to detect bias in political and government terms, (4) showing sensitivity to punctuation and capitalization, as well as (5) struggling with sentences that contain both neutral and gendered subjects.

   Taking everything into account, detecting gender bias in MT remains a challenging task. In morphologically rich languages like German, the subtle linguistic patterns are particularly difficult for a model to learn. Nonetheless, the model's strong performance in practical scenarios, including the recognition of generic masculines and stereotypical role assignments in job postings, demonstrates its efficacy in flagging potential bias for users. This establishes the model as a crucial intermediary layer: it raises awareness by preventing bias from going unnoticed and contributes to mitigating the representational harm that biased translations can reinforce.

\section{Limitations of this work}
   This work is primarily limited by its design choices and the resources that were available. The bias detector analyzes sentences in isolation, which allows users to quickly see the flagged sentences. This design simplifies processing but also means that bias depending on broader context can be missed. For instance, if a text introduces “the doctor is Mr. Smith” and later uses a male form for “the doctor,” this is not actually biased, but the system cannot account for it. Detecting such cases would require word-level analysis and annotated data, which were not available. Performance is also limited by the small size of high-quality datasets. To mitigate this, synthetic data and a handcrafted evaluation set were used, but this introduces the risk that subjective choices propagate into the model. Similarly, the system assumes that translations, whether generated by OpusMT or entered manually, are correct. Mistakes in translation can therefore cause misleading predictions, since translation quality was not taken into account in this work. These constraints are compounded by the model’s sensitivity to sentence-level features. Classification can be influenced by capitalization, punctuation, or multiple subjects in a sentence. Moreover, the lack of interpretability measures prevents the model from explaining why it flagged a sentence as biased, which limits understanding of its predictions and potential corrective actions. Overall, these choices result in a sentence-level tool capable of detecting explicit gender bias, but they limit the system’s ability to handle context, translation errors, and more subtle linguistic phenomena.

\section{Outlook}
   To address these limitations, adding more samples and incorporating natural, word-level labeled data would give the model more diverse examples, allowing it to better capture different patterns of gender bias and handle context-dependent cases. Expanding coverage to additional domains, such as adjective-based biases, would make the system more robust across different types of texts. Moreover, moving beyond binary classification toward token-level analysis could enable the model to identify specific words or phrases responsible for bias and classify them more precisely, for instance as “fair” or “generic masculine.” These improvements would not only enhance the accuracy of bias detection but also provide users with clearer guidance on how translations may perpetuate gendered assumptions.

   The interdisciplinary nature of this task means that progress in ML must be accompanied by advances in social and linguistic research. Understanding these dimensions of GFL helps create more reliable datasets and better ways to measure it. At the same time, exploring alternative model architectures, such as LLM with prompt engineering, could reveal new approaches to bias detection. Ultimately, the long-term goal is for MT systems to handle gendered languages more accurately on their own, reducing the need for intermediary tools. Until that point, applications like the one presented here make bias visible and support more equitable translation practices.

%TC:endinclude
% -----------------------------------
\backmatter 
%TC:ignore
\printbibliography
\clearpage{\pagestyle{empty}\cleardoublepage}
\begingroup
\setcounter{tocdepth}{1}
\addcontentsline{toc}{chapter}{Appendices}		
\chapter*{Appendix}
\addcontentsline{toc}{chapter}{Appendix} 

\begin{figure}
	\centering
	\includegraphics[width=0.7\textwidth]{gt_surgeon_example.png}
	\caption[Example of Google Translate's biased translation]{Google Translate translates an occupational term with a gender stereotype, using the masculine form for "surgeon"}
	\label{fig:gt_surgeon_example}
\end{figure}

\begin{figure}
	\centering
	\includegraphics[width=0.7\textwidth]{deepL_surgeon_example.png}
	\caption[Example of DeepL's biased translation]{DeepL translates the same occupational term with a gender bias, mirroring Google Translate's masculine default for "surgeon"}
	\label{fig:deepL_surgeon_example}
\end{figure}

\begin{figure}
	\centering
	\includegraphics[width=0.7\textwidth]{gt_prates_example.png}
	\caption[Google Translate Gendered Pronoun Suggestions]{Google Translate assigns gendered pronouns in translation for an originally gender-ambiguous subject}
	\label{fig:gt_prates_example}
\end{figure}

\section{Prompt and Output for Pre-training/Fine-tuning Explanation}
\label{appendix:prompt_pdf}
\includepdf[pages=-]{./Literatur/Linear and Softmax 14-jul-2025.pdf}

\section{Handcrafted Test Set}
\label{appendix:handcrafted_testset}

\begin{longtable}{|p{6.5cm}|p{6.5cm}|c|}
\hline
\textbf{English} & \textbf{German} & \textbf{Label} \\
\hline
\multicolumn{3}{|l|}{\textbf{Neutral}} \\
\hline
Today the weather is beautiful & Heute ist das Wetter schön & 0 \\
I like this guy. & Ich mag diesen Kerl. & 0 \\
How are you? & Wie geht es dir? & 0 \\
BERT is a transformers model pretrained on a large corpus of multilingual data in a self-supervised fashion. & BERT ist ein Transformer-Modell, das auf selbstüberwachte Weise auf einem großen Korpus mehrsprachiger Daten vorab trainiert wird. & 0 \\
Knowledge is love and light and vision. & Wissen ist Liebe und Licht und Vision. & 0 \\
\hline
\multicolumn{3}{|l|}{\textbf{Neutral with Gendered Subjects}} \\
\hline
My mother is an engineer. & Meine Mutter ist Ingenieurin. & 0 \\
Is this boy a good artist? & Ist dieser Junge ein guter Künstler? & 0 \\
I am living with my sister, who is also my best friend & Ich lebe mit meiner Schwester, die auch meine beste Freundin ist & 0 \\
My father was an excellent cook. & Mein Vater war ein ausgezeichneter Koch. & 0 \\
The girls went hiking. & Die Mädchen gingen wandern. & 0 \\
\hline
\multicolumn{3}{|l|}{\textbf{Biased}} \\
\hline
Do you like our maths teacher? & Mögen Sie unsere Mathelehrerin? & 1 \\
The doctor was late to work today. & Der Arzt kam heute zu spät zur Arbeit. & 1 \\
Tomorrow the students are leaving for a class trip. & Morgen gehen die Studenten zu einer Klassenfahrt. & 1 \\
This nurse does not work hard. & Diese Krankenschwester arbeitet nicht hart. & 1 \\
Athletes earn a lot of money. & Sportler verdienen viel Geld. & 1 \\
\hline
\multicolumn{3}{|l|}{\textbf{GFL Variants}} \\
\hline
Engineers are kind and hard-working & Ingenieur*innen sind lieb und fleißig & 0 \\
Engineers are kind and hard-working & Ingenieure und Ingenieurinnen sind lieb und fleißig & 0 \\
Teachers are kind and hard-working & Lehrende sind lieb und fleißig & 0 \\
Teachers are kind and hard-working & Lehrer:innen sind lieb und fleißig & 0 \\
Teachers are kind and hard-working & Lehrerinnen und Lehrer sind lieb und fleißig & 0 \\
Teachers are kind and hard-working & Lehrer sind lieb und fleißig & 1 \\
Teachers are kind and hard-working & Lehrerinnen sind lieb und fleißig & 1 \\
\hline
\multicolumn{3}{|l|}{\textbf{Job Posting (Real-world)} } \\
\hline
We’re seeking someone to join our team Office 365 squads to lead the design, development, and integration of Gen AI apps and integration using Microsoft Copilot Studio. & Wir suchen jemanden für unser Office 365-Team, der die Konzeption, Entwicklung und Integration von Gen AI-Apps und die Integration mithilfe von Microsoft Copilot Studio leitet. & 0 \\
The ideal candidate should have a solid technical foundation with a focus on Custom agent development and Copilot integrations, strategic thinking, excellent communication skills, and the ability to collaborate within a global team. & Der ideale Kandidat sollte über solide technische Grundlagen mit Schwerpunkt auf der Entwicklung kundenspezifischer Agenten und Copilot-Integrationen, strategisches Denken, ausgezeichnete Kommunikationsfähigkeiten und die Fähigkeit zur Zusammenarbeit in einem globalen Team verfügen. & 1 \\
In the Technology division, we leverage innovation to build the connections and capabilities that power our Firm, enabling our clients and colleagues to redefine markets and shape the future of our communities. & Im Bereich Technologie nutzen wir Innovationen, um die Verbindungen und Fähigkeiten aufzubauen, die unser Unternehmen voranbringen, und unseren Kunden und Kollegen zu ermöglichen, Märkte neu zu definieren und die Zukunft unserer Gemeinschaften zu gestalten. & 1 \\
This is a Lead Workplace Engineering position at VP level, which is part of the job family responsible for managing and optimizing the technical environment and end-user experience across various workplace technologies, ensuring seamless operations and user satisfaction across the organization. & Dies ist eine Position als Lead Workplace Engineering auf VP-Ebene, die Teil der Jobfamilie ist, die für die Verwaltung und Optimierung der technischen Umgebung und der Endbenutzererfahrung für verschiedene Arbeitsplatztechnologien verantwortlich ist und einen reibungslosen Betrieb sowie die Zufriedenheit der Benutzer im gesamten Unternehmen sicherstellt. & 1 \\
\hline
\caption[Handcrafted EN-DE sentence pairs]{Handcrafted EN-DE sentence pairs with binary bias labels (0 = neutral, 1 = biased)}
\end{longtable}
\endgroup
\newpage
\thispagestyle{empty}
\noindent

1. Hiermit versichere ich,
\begin{itemize}
    \item dass ich die von mir vorgelegte Arbeit selbständig abgefasst habe,
    \item dass ich keine weiteren Hilfsmittel verwendet habe als diejenigen, die im Vorfeld explizit zugelassen und von mir angegeben wurden,
    \item dass ich die Stellen der Arbeit, die dem Wortlaut oder dem Sinn nach anderen Werken (dazu zählen auch Internetquellen und KI-basierte Tools) entnommen sind, unter Angabe der Quelle kenntlich gemacht habe und
    \item dass ich die vorliegende Arbeit noch nicht für andere Prüfungen eingereicht habe.
\end{itemize}
2. Mir ist bewusst,
\begin{itemize}
    \item dass ich diese Prüfung nicht bestanden habe, wenn ich die mir bekannte Frist für die Einreichung meiner schriftlichen Arbeit versäume,
    \item dass ich im Falle eines Täuschungsversuchs diese Prüfung nicht bestanden habe,
    \item dass ich im Falle eines schwerwiegenden Täuschungsversuchs ggf. die Gesamtprüfung endgültig nicht bestanden habe und in diesem Studiengang nicht mehr weiter studieren darf und
    \item dass ich, sofern ich zur Erstellung dieser Arbeit KI-basierter Tools verwendet habe, die Verantwortung für eventuell durch die KI generierte fehlerhafte oder verzerrte (bias) Inhalte, fehlerhafte Referenzen, Verstöße gegen das Datenschutz- und Urheberrecht oder Plagiate trage.
\end{itemize}

\vspace{2cm}

\noindent
Berlin, den \today

\vspace{3cm}

\hspace*{7cm}%
\dotfill\\
\hspace*{8.5cm}%
\textit{(Unterschrift des Verfassers)}
 			% Eidesstattliche Erklärung (nicht bei Seminararb.)
%TC:endignore

\end{document}