\documentclass[12pt,twoside=off,
bibliography=totoc,listof=totoc,appendixprefix,paper=a4,headings=small]{scrbook} % 'twoside=on' für Druckversion oder 'twoside=off' für Onlineversion

%
% Packages
% -----------------------------------
\usepackage[
  paper=a4paper,
  left=12.5mm,
  right=25mm,
  top=25mm,
  bottom=50mm,
  bindingoffset=10mm]{geometry}		% Seitenränder und Bindungskorrektur einstellen
  
\usepackage{natbib}		

\setcitestyle{round,aysep={}} 		% Indizierg. in runden Klammern, zw. Autor u. Jahr
\usepackage[utf8]{inputenc}         % Umlaute im Text
\usepackage[T1]{fontenc}
\usepackage{lmodern}				% Schriftfamilie
\setkomafont{disposition}{\bfseries}

\usepackage{graphicx} 				% Grafiken einfügen (pdf,png - aber jpg vermeiden)
\graphicspath{{./Bilder/}}          % Pfad zu den Bildern

\usepackage[final,nopatch=footnote]{microtype}

\usepackage{url}					% URL's formatieren (z.B. in Literatur) 
\usepackage[colorlinks,linkcolor=black,citecolor=black,urlcolor=black]{hyperref} 				% für Hyperlinks in PDF-Dokumenten   
  
\usepackage{tabularx} 				% bessere Gestaltung von Tabellen
\usepackage{longtable} 		
\usepackage{multicol}				
\usepackage{multirow}
\usepackage{booktabs}
\usepackage{tabularx}
		
\usepackage[active]{srcltx}

\usepackage{listings}				% Algorithmen

\usepackage{mdwlist}				% Listen

\usepackage{setspace} 				% Zeileneinstellung
\newtheorem{mydef}{Merksatz}  		% Falls Beispiele, Merksätze m. fortl. Nr. gebr. werden
\newtheorem{bsp}{Beispiel}

\usepackage{lscape}					% zum Rotieren von Seiten

\usepackage{amsmath}				% zum Schreiben von mathematischen Formeln

\usepackage{calc}

\usepackage{footnote}				% Fußnoten
\usepackage{tablefootnote}			% Fußnoten in Tabellen

%\clubpenalty = 10000
%\widowpenalty = 10000 \displaywidowpenalty = 10000

\hyphenation{voll-st\"andigen}		% Worttrennungen global definieren

\setcounter{tocdepth}{3}			% Ebenen, die im Inhaltsverzeichnis angezeigt werden

% Document
% -----------------------------------
\begin{document}

\frontmatter 
    % Titelseite soll keine Kopf oder Fußzeile haben
\thispagestyle{empty}

% Alle Elemente sollen zentriert sein
\begin{center}

\vspace*{-10mm}
\includegraphics[width=0.45\textwidth]{Hochschule_für_Wirtschaft_und_Recht_Berlin_logo.svg.png}\\[10mm]

Berlin School of Economics and Law\\
Department I - Business and Economics\\[15mm]

{\Large \textbf{Detecting Gender Bias in}}\\ 
\vspace*{2mm}
{\Large \textbf{English-German Translations}}\\ 
\vspace*{2mm}
{\Large \textbf{using Natural Language Processing}}\\

\vspace*{\fill} 

{\LARGE {Bachelor's Thesis}}\\ 

\vspace*{\fill} 

for the attainment of the academic degree Bachelor of Science (B.Sc.)\\ \vspace*{1.5mm} 
in the study program\\\vspace*{1.5mm}
\textbf{Information Systems Management}\\\vspace*{1.5mm}


\vspace*{\fill} 

% Name des/der Autors/Autoren
{\Large Submitted by Khanh Linh Pham}\\[15mm]

\vspace*{\fill} 

% Gutachter, Kontaktdaten und Abgabetermin
\begin{flushleft}
\begin{tabbing}
Main Supervisor:\hspace{1.6cm} \= Prof. Dr. Diana Hristova \\
Secondary Supervisor:\> Prof. Dr. Markus Schaal \\[4mm]
Semester:\> Summer Semester 2025\\
Matriculation no.:\> 77211916753\\
Email:\> klpham04@gmail.com\\[8mm]
\textbf{Date of Submission:} \> \textbf{September 01, 2025}\\
\end{tabbing}
\end{flushleft}

\end{center}

\clearpage{\pagestyle{empty}\cleardoublepage} 			% Titelblatt
    \clearpage{\pagestyle{empty}\cleardoublepage}
    \thispagestyle{empty}


\vspace*{1cm}

\begin{center}
    \textbf{Abstract}
\end{center}

\vspace*{1cm}

\noindent 
Gender bias in English–German Machine Translation often appears in forms such as generic masculine defaulting and occupation stereotyping. These biases can perpetuate unequal representations and feed back into future translation models, reinforcing biased outputs in society. This thesis examines how accurately multilingual BERT (mBERT) can detect such bias. The model was fine-tuned on limited datasets with varying annotation quality, which caused its main limitations. The classifier occasionally (1) misclassifies German gender-fair language forms as biased, (2) fails to detect bias in political and government terms, (3) fails to recognize semantically gendered words as unbiased, (4) is sensitive to punctuation and capitalization, and (5) struggles with sentences that contain both neutral and gendered subjects. Despite these gaps, the model achieved an F1 score of 0.966 and proves effective for core bias cases. It reached 84.6\% accuracy on a small handcrafted evaluation dataset with practical sentences like job postings and edge cases. As an intermediary step, the work offers a trained model, sufficiently effective for practical bias detection, and an application that make biased translations visible while indicating areas for further investigation and improvement. The code is available at \url{https://github.com/phmkhali/bias-detector-en-de}.
    \newpage
    
    \clearpage{\pagestyle{empty}\cleardoublepage}
    \thispagestyle{empty}


\vspace*{1cm}

\begin{center}
    \textbf{Sperrvermerk}
\end{center}

\vspace*{1cm}

\noindent 

XX

\vspace{2cm}

\noindent
Berlin, den 01. Januar 2099

\vspace{3cm}

\hspace*{7cm}%
\dotfill\\
\hspace*{8.5cm}%
\textit{(Unterschrift des Verfassers)}      % Sperrvermerk
    \newpage
    
    \clearpage{\pagestyle{empty}\cleardoublepage}
    \onehalfspacing                  	% Zeilenabstand ab hier 1.5 zeilig
    \tableofcontents 					% Inhaltsverzeichnis
    \clearpage{\pagestyle{empty}\cleardoublepage} 
    
    \listoffigures 					 	% Abbildungsverzeichnis
    \clearpage{\pagestyle{empty}\cleardoublepage}
    
    \listoftables						% Tabellenverzeichnis
    \clearpage{\pagestyle{empty}\cleardoublepage}

% -----------------------------------
\mainmatter 							% die einzelnen Kapitel
    \chapter{Introduction}
Machine Translation (MT) helps millions of people communicate across languages, in daily life and in areas like healthcare, law, and business \citep{kapplAreAllSpanish2025}. Services like Google Translate handle over 200 million users every day \citep{pratesAssessingGenderBias2019,shresthaExploringGenderBiases2022}. It is a fast-growing market. A report by \citet{skyquestMachineTranslationMT2025} valued it at 980 million USD in 2023, with projections reaching 2.78 billion USD. New and more advanced translation models keep appearing, and many of them are free to use. As a result, MT tools are now used to translate large volumes of content across domains.

With this widespread use, the output of MT systems increasingly shapes how people receive and interpret information. But automatic translations are not neutral. There is growing concern about the social effects of biased translations. One key issue is gender bias. MT systems are often trained on large datasets that reflect social norms and stereotypes. If the data contains gender bias, the system will likely reproduce it \citep{choMeasuringGenderBias2019,soundararajanInvestigatingGenderBias2024,smacchiaDoesAIReflect2024}.

A common case is the use of gendered terms in translations of gender-neutral input. For example, the English sentence “The nurse is hard-working” does not say anything about gender. But a translation system may render it in German as “Die Krankenschwester ist fleißig,” which uses the explicitly feminine term \textit{Krankenschwester}. Similarly, “The surgeon is hard-working” may become “Der Chirurg ist fleißig,” using the masculine form \textit{Chirurg}. These choices add gendered assumptions that were not present in the original. Such patterns are not just technical side effects. They can reinforce stereotypes, especially when they appear in job ads, reports, or other public texts.


\section{Motivation}

\subsection{Social and Ethical Importance of Addressing Gender Bias}\label{section:social_and_ethical_importance_of_addressing}
Academia has come to the consensus that MT systems do default to male pronouns when gender in the source sentence is ambiguous \citep{pratesAssessingGenderBias2019,choMeasuringGenderBias2019,rescignoGenderBiasMachine2023}.  In addition, translations often reflect traditional roles, like associating “nurse” with women and “surgeon” with men. This can affect people’s perceptions of jobs and reinforce gender roles.

When used in formal contexts like job descriptions or reference letters, biased translations can shape how a candidate is perceived. If a system always assigns male pronouns to leadership roles and female terms to caregiving roles, it may disadvantage those who do not match those stereotypes \citep{bolukbasiManComputerProgrammer2016}. This is not just a personal issue. It can reduce diversity and go against international standards. Organizations like the United Nations, UNESCO, and the European Union stress the importance of gender equality and inclusive language, making gender equality one of the 17 Sustainable Development Goals for 2030 \citep{sczesnyCanGenderFairLanguage2016,unitednationsAchieveGenderEquality2023}. 

Language also shapes thought. Research shows that readers often interpret masculine forms as male-specific, even if they are supposed to be generic \citep{sczesnyCanGenderFairLanguage2016}. Inclusive forms are more common in official documents, less so in everyday language. However, exposure matters. Frequent use of fair language makes it feel more normal. Detecting and addressing bias in MT can support this shift.

\subsection{Why Detection Systems Are Needed}

Current research on this topic tends to focus more on the quantitative measurement of gender bias \citep{rescignoGenderBiasMachine2023,barclayInvestigatingMarkersDrivers2024a,smacchiaDoesAIReflect2024}. Common methods include counting gendered forms in outputs and comparing them to demographic baselines or human expectations \citep{rescignoGenderBiasMachine2023,pratesAssessingGenderBias2019,savoldiWhatHarmQuantifying2024}. These are useful, but they do not help users identify specific biased translations in real-time. Evaluations are not enough for accountability. 

Other domains, like facial recognition, have already seen progress in active bias detection. For example, \citet{schwemmerDiagnosingGenderBias2020} showed that systems tend to label women more accurately if they match stereotypical appearances (e.g., long hair). Some models even linked female images to words like “kitchen” or “cake” based on bias patterns in training data. For MT, a detection layer is still missing. Without such tools, biased translations are likely to spread unnoticed. A detection system could flag potential bias in real time, improving transparency and encouraging more careful use.

\section{Problem Statement and Research Questions}
\textbf{DRAFT NEED TO REWRITE AFTER IMPLEMENTATION}
This thesis focuses on gender bias in English-to-German (EN-DE) MT. This language pair is widely used in research, with many open datasets and high-quality models available. It also involves a grammatical shift: English has limited gender marking, while German assigns gender to many nouns and pronouns. This structural difference makes gender bias more visible and easier to study in the translation outputs.

The core problem boils down to the significant bias towards the masculine form in EN-DE MTs, sometimes consituting 93-96\% of translations for isolated words \citep{lardelliBuildingBridgesDataset2024}. These outputs often reflect social stereotypes rather than objective translations, yet current systems offer no mechanism to detect or signal when such bias occurs \citep{rescignoGenderBiasMachine2023}. To address this, this thesis deploys a blackbox approach to explore how fine-tuning a pre-trained multilingual BERT model can help detect gender bias in MT outputs. The model takes an input sentence and its corresponding German translation and predicts whether the translation introduces gender bias. 

The translation system used is \href{https://github.com/Helsinki-NLP/Opus-MT?tab=readme-ov-file}{Opus-MT}, an open-source neural MT model. It is widely used in research, supports EN-DE translation, and is trained on real-world corpora, making it suitable for studying translation bias \citep{tiedemannOPUSMTBuildingOpen2020}. Translations are then passed through BERT, trained on a dataset I have constructed by combining and adapting several existing datasets from other researchers. The classifier is lightweight and efficient, aiming for transparent behavior and easy integration into other tools \citep{devlinBERTPretrainingDeep2019}. The final tool highlights biased parts in a simple web demo. The goal is not a perfect classifier but a working prototype that shows how such detection could be integrated into translation workflows.

The main research question is therefore: \textbf{"How can a NLP-based binary classification model detect gender bias in English-German translations?"}. 

\section{Scope}

\textbf{WRITE AFTER IMPLEMENTATION PART}
This thesis focuses only on EN-DE MT. Other language pairs are out of scope.

\section{Limitations}
\textbf{WRITE AFTER IMPLEMENTATION PART}
It becomes especially difficult to detect when sentences contain multiple subjects, indirect references, or ambiguous pronouns. For example, as \citet{barclayInvestigatingMarkersDrivers2024a} explain, the sentence “He went to see her mother” clearly implies three people, while “He went to see his mother” could refer to either two or three. These types of structures introduce ambiguity that makes annotation and evaluation much harder. Creating a dataset that captures such linguistic complexity would require significant effort and careful control of variables. One broader limitation in building datasets for complex scenarios with multiple subjects is the difficulty of isolating the influence of each gendered entity \citep{lardelliBuildingBridgesDataset2024}. When working with natural language sources, it becomes hard to tell what caused the bias in the translation. Because of this, the focus of this thesis is on simpler sentence structures with a single subject. This makes it easier to identify and explain bias patterns. It also fits the intended use case: translating business texts like job advertisements or reports, which rarely involve multiple nested clauses or ambiguous pronouns.
 

\section{Overview of Chapters}
\textbf{WRITE AFTER IMPLEMENTATION PART}

    \clearpage{\pagestyle{empty}\cleardoublepage}		
    \chapter{Related Work}

This section outlines key findings of related work on gender bias in MT, with a focus on the English-German language pair. The research aims are to (1) define the core concept of gender bias in MT, (2) establish the relevance of the topic, (3) identify the research gap, and (4) justify technical design choices. To support this, I examine datasets, model types, and tools used in previous studies.

For the literature review I combined incremental and conceptual literature review methods, where each source led to the identification of the next. Based on this progression, I identified key concepts and used them to organize and interpret the literature, aligning with a conceptual approach. The structure followed the qualitative Information Systems framework by \citet{schryenWritingQualitativeLiterature2015} and further informed by \citet{shresthaExploringGenderBiases2022}, who conducted a systematic review on gender bias in ML and AI. 

\section{Literature Search Process}

\subsection{Search Sources and Tools}
Sources were primarily searched on \href{https://scholar.google.com/}{Google Scholar} and \href{https://www.perplexity.ai/}{Perplexity}, which served as an additional search engine. Prompts and outputs from Perplexity have been saved and are included in the appendix. To organize and manage the collected sources, \href{https://www.zotero.org/}{Zotero} was used throughout the process.

\subsection{Literature Review Framing}

To answer the four research aims, I have defined the key concepts in \autoref{tab:key-concepts}. Key search terms consisted of \textit{gender bias}, \textit{machine translation}, \textit{AI}, \textit{machine learning}, \textit{German}, \textit{stereotypes}, and \textit{detection}. The focus was on literature published between 2019 and 2025 to maintain relevance and currency, while foundational and definitional works from earlier periods were selectively included. The initial search for the term \textit{gender bias in machine translation} returned over 18,000 results. Through my iterative selection process, this was narrowed down to 34 core sources.

\renewcommand{\arraystretch}{1.3}
\begin{table}[ht!]
\centering
\begin{tabularx}{\textwidth}{lX}
\toprule
\textbf{Key Concept} & \textbf{Description} \\
\midrule
Stereotypes and Biases in Society & Introduces the social foundations of bias by explaining how stereotypes form, persist, and shape expectations about gender roles. Necessary to understand why certain translation outputs may reflect or reinforce societal gender norms. \\

Machine and Algorithmic Bias & Explains how social biases can enter ML systems through data, design choices, or feedback mechanisms. Sets the groundwork for understanding how gender bias can emerge in translation models used in this thesis. \\

Gender Bias in English-German Translation & Focuses on the specific challenges of translating between English and German, where the lack of grammatical gender in English and its necessity in German can cause biased outputs. Defines the types of gender bias relevant to the classification task in this thesis. \\

Bias Detection Frameworks & Reviews existing methods for identifying gender bias in language data and ML outputs. Helps justify the choice of a classification approach for detecting bias in translations. \\

\bottomrule
\end{tabularx}
\caption{Key concepts relevant to this thesis}
\label{tab:key-concepts}
\end{table}

\subsection{Citation Tracking}
Backward citation searching involved reviewing references cited by selected papers, prioritizing frequently cited and foundational works relevant to gender bias in MT. Forward citation searching used Google Scholar's "cited by" function to identify newer research citing those key papers. Filtering with specific terms (e.g., \textit{German} and \textit{machine translation}) was applied during forward search to maintain focus. In addition to the main review process, supplementary sources were included as needed throughout the writing phase. These consist of contextual references, statistics, or secondary citations that support specific points but were not part of the core conceptual or methodological framework.

\subsection{Selection Criteria and Screening Process}
Titles and abstracts were manually screened to select relevant studies. Inclusion criteria required sources to specifically address gender bias in MT, provide examples or discussions of gender-related errors, or explain the significance of gender bias in this context. Exclusion criteria filtered out studies focusing on general NLP bias without a direct link to MT, non-gender biases without clear gender connection, and highly technical papers lacking contribution to the general understanding of gender bias. Full texts were reviewed after initial screening to confirm relevance and extract insights. Redundant sources not providing new perspectives aligned with the thesis goals were excluded.

\section{Empirical Evidence}
% say that it is well documented in MT
% Summarize how researchers measured bias (benchmarks, manual annotation, templates).


\section{Gender Bias in NLP}
% how it manifests
% how it affects it in this specific case
% refer to bias because of society
% how context affects it

\section{Other Approaches}
% compare how it has been approached with other languages
% what experiements they used and what systems
% GERMAN ENGLISH PAPERS
% - Overview of rule-based, ML-based, and LLM-based approaches.
    
\section{Limitations Identified by Prior Research}    


\section{Positioning of My Work}    
% - How your work builds on these findings.
% - What gap you address (binary classifier + demo).
    
%     **Use:** Comparison with prior methods.
    \clearpage{\pagestyle{empty}\cleardoublepage}		
    \chapter{Conceptual Frameowrk}

\section{What is Gender Bias?}
% what is bias
% what is gender bias - basics
% types of biases, what counts as it
% statistics and prove how it affects people

\section{What is Machine Bias?}
% what is bias in NLP - basics
% how training data affects bias generally
% challenges

\section{Differences between the English and German Language}
% explain differences in language structures, specific examples
% grammatical vs natural framework
% german gfl
% strategies
% debates
% what i will use

\section{Binary Classification in NLP}
% What binary classification is
% Why it fits your task (biased vs. not biased)
% Common algorithms used (just mention; detail comes in practical part)
% How input features are represented (e.g. embeddings)

\section{Pre-trained Language Model: BERT}
% what is bert
% how it works
% how i am planning to use it


% -----------------------------------
\backmatter 
\bibliographystyle{apalike}		
\bibliography{./Literatur/refs}		% Literaturquellen einbinden 
\clearpage{\pagestyle{empty}\cleardoublepage}		
\chapter*{Appendix}
\addcontentsline{toc}{chapter}{Appendix} 

\begin{figure}
	\centering
	\includegraphics[width=0.7\textwidth]{gt_surgeon_example.png}
	\caption[Example of Google Translate's biased translation]{Google Translate translates an occupational term with a gender stereotype, using the masculine form for "surgeon"}
	\label{fig:gt_surgeon_example}
\end{figure}

\begin{figure}
	\centering
	\includegraphics[width=0.7\textwidth]{deepL_surgeon_example.png}
	\caption[Example of DeepL's biased translation]{DeepL translates the same occupational term with a gender bias, mirroring Google Translate's masculine default for "surgeon"}
	\label{fig:deepL_surgeon_example}
\end{figure}

\begin{figure}
	\centering
	\includegraphics[width=0.7\textwidth]{gt_prates_example.png}
	\caption[Google Translate Gendered Pronoun Suggestions]{Google Translate assigns gendered pronouns in translation for an originally gender-ambiguous subject}
	\label{fig:gt_prates_example}
\end{figure}

\section{Prompt and Output for Pre-training/Fine-tuning Explanation}
\label{appendix:prompt_pdf}
\includepdf[pages=-]{./Literatur/Linear and Softmax 14-jul-2025.pdf}

\section{Handcrafted Test Set}
\label{appendix:handcrafted_testset}

\begin{longtable}{|p{6.5cm}|p{6.5cm}|c|}
\hline
\textbf{English} & \textbf{German} & \textbf{Label} \\
\hline
\multicolumn{3}{|l|}{\textbf{Neutral}} \\
\hline
Today the weather is beautiful & Heute ist das Wetter schön & 0 \\
I like this guy. & Ich mag diesen Kerl. & 0 \\
How are you? & Wie geht es dir? & 0 \\
BERT is a transformers model pretrained on a large corpus of multilingual data in a self-supervised fashion. & BERT ist ein Transformer-Modell, das auf selbstüberwachte Weise auf einem großen Korpus mehrsprachiger Daten vorab trainiert wird. & 0 \\
Knowledge is love and light and vision. & Wissen ist Liebe und Licht und Vision. & 0 \\
\hline
\multicolumn{3}{|l|}{\textbf{Neutral with Gendered Subjects}} \\
\hline
My mother is an engineer. & Meine Mutter ist Ingenieurin. & 0 \\
Is this boy a good artist? & Ist dieser Junge ein guter Künstler? & 0 \\
I am living with my sister, who is also my best friend & Ich lebe mit meiner Schwester, die auch meine beste Freundin ist & 0 \\
My father was an excellent cook. & Mein Vater war ein ausgezeichneter Koch. & 0 \\
The girls went hiking. & Die Mädchen gingen wandern. & 0 \\
\hline
\multicolumn{3}{|l|}{\textbf{Biased}} \\
\hline
Do you like our maths teacher? & Mögen Sie unsere Mathelehrerin? & 1 \\
The doctor was late to work today. & Der Arzt kam heute zu spät zur Arbeit. & 1 \\
Tomorrow the students are leaving for a class trip. & Morgen gehen die Studenten zu einer Klassenfahrt. & 1 \\
This nurse does not work hard. & Diese Krankenschwester arbeitet nicht hart. & 1 \\
Athletes earn a lot of money. & Sportler verdienen viel Geld. & 1 \\
\hline
\multicolumn{3}{|l|}{\textbf{GFL Variants}} \\
\hline
Engineers are kind and hard-working & Ingenieur*innen sind lieb und fleißig & 0 \\
Engineers are kind and hard-working & Ingenieure und Ingenieurinnen sind lieb und fleißig & 0 \\
Teachers are kind and hard-working & Lehrende sind lieb und fleißig & 0 \\
Teachers are kind and hard-working & Lehrer:innen sind lieb und fleißig & 0 \\
Teachers are kind and hard-working & Lehrerinnen und Lehrer sind lieb und fleißig & 0 \\
Teachers are kind and hard-working & Lehrer sind lieb und fleißig & 1 \\
Teachers are kind and hard-working & Lehrerinnen sind lieb und fleißig & 1 \\
\hline
\multicolumn{3}{|l|}{\textbf{Job Posting (Real-world)} } \\
\hline
We’re seeking someone to join our team Office 365 squads to lead the design, development, and integration of Gen AI apps and integration using Microsoft Copilot Studio. & Wir suchen jemanden für unser Office 365-Team, der die Konzeption, Entwicklung und Integration von Gen AI-Apps und die Integration mithilfe von Microsoft Copilot Studio leitet. & 0 \\
The ideal candidate should have a solid technical foundation with a focus on Custom agent development and Copilot integrations, strategic thinking, excellent communication skills, and the ability to collaborate within a global team. & Der ideale Kandidat sollte über solide technische Grundlagen mit Schwerpunkt auf der Entwicklung kundenspezifischer Agenten und Copilot-Integrationen, strategisches Denken, ausgezeichnete Kommunikationsfähigkeiten und die Fähigkeit zur Zusammenarbeit in einem globalen Team verfügen. & 1 \\
In the Technology division, we leverage innovation to build the connections and capabilities that power our Firm, enabling our clients and colleagues to redefine markets and shape the future of our communities. & Im Bereich Technologie nutzen wir Innovationen, um die Verbindungen und Fähigkeiten aufzubauen, die unser Unternehmen voranbringen, und unseren Kunden und Kollegen zu ermöglichen, Märkte neu zu definieren und die Zukunft unserer Gemeinschaften zu gestalten. & 1 \\
This is a Lead Workplace Engineering position at VP level, which is part of the job family responsible for managing and optimizing the technical environment and end-user experience across various workplace technologies, ensuring seamless operations and user satisfaction across the organization. & Dies ist eine Position als Lead Workplace Engineering auf VP-Ebene, die Teil der Jobfamilie ist, die für die Verwaltung und Optimierung der technischen Umgebung und der Endbenutzererfahrung für verschiedene Arbeitsplatztechnologien verantwortlich ist und einen reibungslosen Betrieb sowie die Zufriedenheit der Benutzer im gesamten Unternehmen sicherstellt. & 1 \\
\hline
\caption[Handcrafted EN-DE sentence pairs]{Handcrafted EN-DE sentence pairs with binary bias labels (0 = neutral, 1 = biased)}
\end{longtable}
\newpage
\thispagestyle{empty}
\noindent

1. Hiermit versichere ich,
\begin{itemize}
    \item dass ich die von mir vorgelegte Arbeit selbständig abgefasst habe,
    \item dass ich keine weiteren Hilfsmittel verwendet habe als diejenigen, die im Vorfeld explizit zugelassen und von mir angegeben wurden,
    \item dass ich die Stellen der Arbeit, die dem Wortlaut oder dem Sinn nach anderen Werken (dazu zählen auch Internetquellen und KI-basierte Tools) entnommen sind, unter Angabe der Quelle kenntlich gemacht habe und
    \item dass ich die vorliegende Arbeit noch nicht für andere Prüfungen eingereicht habe.
\end{itemize}
2. Mir ist bewusst,
\begin{itemize}
    \item dass ich diese Prüfung nicht bestanden habe, wenn ich die mir bekannte Frist für die Einreichung meiner schriftlichen Arbeit versäume,
    \item dass ich im Falle eines Täuschungsversuchs diese Prüfung nicht bestanden habe,
    \item dass ich im Falle eines schwerwiegenden Täuschungsversuchs ggf. die Gesamtprüfung endgültig nicht bestanden habe und in diesem Studiengang nicht mehr weiter studieren darf und
    \item dass ich, sofern ich zur Erstellung dieser Arbeit KI-basierter Tools verwendet habe, die Verantwortung für eventuell durch die KI generierte fehlerhafte oder verzerrte (bias) Inhalte, fehlerhafte Referenzen, Verstöße gegen das Datenschutz- und Urheberrecht oder Plagiate trage.
\end{itemize}

\vspace{2cm}

\noindent
Berlin, den \today

\vspace{3cm}

\hspace*{7cm}%
\dotfill\\
\hspace*{8.5cm}%
\textit{(Unterschrift des Verfassers)}
 			% Eidesstattliche Erklärung (nicht bei Seminararb.)

\end{document}