\chapter{Introduction}
Machine translation (MT) is a sub-field of computational linguistic that uses computer software to translate texts between languages \citep{linMachineTranslationAcademic2009}. It is a major area within Natural Language Processing (NLP), a branch of Artificial Intelligence (AI) \citep{smacchiaDoesAIReflect2024}. Millions of people rely on MT systems daily to overcome language barriers in various critical domains such as medicine, business, and diplomacy \citep{kapplAreAllSpanish2025}. Tools like Google Translate serve over 200 million users daily \citep{pratesAssessingGenderBias2019,shresthaExploringGenderBiases2022}. According to a market analysis by \citet{skyquestMachineTranslationMT2025}, the MT market size was valued at 980 million USD in 2023 and is projected to reach 2.78 billion USD by 2023. 

With the growing availability of free MT tools capable of handling complex sentences, their use in translating large volumes of online content is increasing, therefore expanding their influence on global access to information. Machine-generated content is especially common in lower-resource languages, where it often makes up a large share of all available online content \citep{thompsonShockingAmountWeb2024}. This large-scale automated translation raises new concerns: not just about quality, but also about bias. One aspect is gender bias. Several studies \citep{smacchiaDoesAIReflect2024,choMeasuringGenderBias2019,stanczakSurveyGenderBias2021,soundararajanInvestigatingGenderBias2024} confirm that MT systems, trained on large-scale corpora that incorporate societal biases, can learn and perpetuate gender stereotypes and biases present in the training data. This includes biases related to occupations, associating gender-neutral terms with gender-specified pronouns.

When translating between languages with and without grammatical gender, there is a risk of incorrect gender assignment. For example, the English sentences "The surgeon is hard-working" and "The nurse is hard-working" are translated into German as "Der Chirurg ist fleißig" and "Die Krankenschwester ist fleißig" respectively, as seen in \autoref{fig:gt_surgeon_example} and \autoref{fig:deepL_surgeon_example}. These translations introduce occupational gender stereotypes. “Der Chirurg” is the masculine form of “surgeon” in German, adding a male gender where the original English sentence did not specify one. Similarly, “Die Krankenschwester” is the feminine form of “nurse,” again assigning a gender that was not present in the source. These patterns are not just technical flaws. They can reinforce harmful stereotypes in real-world contexts. When MT is used for job descriptions, recommendation letters, or resumes \citep{bolukbasiManComputerProgrammer2016} and it inserts or reinforces unfairly gendered language, it may deter women from applying or reduce their chances of success in recruitment processes. Therefore, addressing gender bias in MT becomes both a social and ethical necessity. The following section outlines the broader motivation behind this thesis.


\section{Motivation}

Vestibulum fringilla pede sit amet augue. In turpis. Pellentesque posuere. Praesent turpis. Aenean posuere, tortor sed cursus feugiat, nunc augue blandit nunc, eu sollicitudin urna dolor sagittis lacus. Donec elit libero, sodales nec, volutpat a, suscipit non, turpis. Nullam sagittis. Suspendisse pulvinar, augue ac venenatis condimentum, sem libero volutpat nibh, nec pellentesque velit pede quis nunc. Vestibulum ante ipsum primis in faucibus orci luctus et ultrices posuere cubilia Curae; Fusce id purus. Ut varius tincidunt libero. Phasellus dolor. Maecenas vestibulum mollis diam. Pellentesque ut neque. Pellentesque habitant morbi tristique senectus et netus et malesuada fames ac turpis egestas. In dui magna, posuere eget, vestibulum et, tempor auctor, justo. 

In ac felis quis tortor malesuada pretium. Pellentesque auctor neque nec urna. Proin sapien ipsum, porta a, auctor quis, euismod ut, mi. Aenean viverra rhoncus pede. Pellentesque habitant morbi tristique senectus et netus et malesuada fames ac turpis egestas. Ut non enim eleifend felis pretium feugiat. Vivamus quis mi. Phasellus a est. Phasellus magna. In hac habitasse platea dictumst. Curabitur at lacus ac velit ornare lobortis. Curabitur a felis in nunc fringilla tristique. 

\section{Problem Statement and Research Questions}

Vestibulum fringilla pede sit amet augue. In turpis. Pellentesque posuere. Praesent turpis. Aenean posuere, tortor sed cursus feugiat, nunc augue blandit nunc, eu sollicitudin urna dolor sagittis lacus. Donec elit libero, sodales nec, volutpat a, suscipit non, turpis. Nullam sagittis. Suspendisse pulvinar, augue ac venenatis condimentum, sem libero volutpat nibh, nec pellentesque velit pede quis nunc. Vestibulum ante ipsum primis in faucibus orci luctus et ultrices posuere cubilia Curae; Fusce id purus. Ut varius tincidunt libero. Phasellus dolor. Maecenas vestibulum mollis diam. Pellentesque ut neque. Pellentesque habitant morbi tristique senectus et netus et malesuada fames ac turpis egestas. In dui magna, posuere eget, vestibulum et, tempor auctor, justo. In ac felis quis tortor malesuada pretium. Pellentesque auctor neque nec urna. Proin sapien ipsum, porta a, auctor quis, euismod ut, mi. Aenean viverra rhoncus pede. Pellentesque habitant morbi tristique senectus et netus et malesuada fames ac turpis egestas. Ut non enim eleifend felis pretium feugiat. Vivamus quis mi. Phasellus a est. Phasellus magna. In hac habitasse platea dictumst. Curabitur at lacus ac velit ornare lobortis. Curabitur a felis in nunc fringilla tristique.

\section{Aufbau der Arbeit}

Vestibulum fringilla pede sit amet augue. In turpis. Pellentesque posuere. Praesent turpis. Aenean posuere, tortor sed cursus feugiat, nunc augue blandit nunc, eu sollicitudin urna dolor sagittis lacus. Donec elit libero, sodales nec, volutpat a, suscipit non, turpis. Nullam sagittis. Suspendisse pulvinar, augue ac venenatis condimentum, sem libero volutpat nibh, nec pellentesque velit pede quis nunc. Vestibulum ante ipsum primis in faucibus orci luctus et ultrices posuere cubilia Curae; Fusce id purus. Ut varius tincidunt libero. Phasellus dolor. Maecenas vestibulum mollis diam. Pellentesque ut neque. Pellentesque habitant morbi tristique senectus et netus et malesuada fames ac turpis egestas. In dui magna, posuere eget, vestibulum et, tempor auctor, justo. In ac felis quis tortor malesuada pretium. Pellentesque auctor neque nec urna. Proin sapien ipsum, porta a, auctor quis, euismod ut, mi. Aenean viverra rhoncus pede. Pellentesque habitant morbi tristique senectus et netus et malesuada fames ac turpis egestas. Ut non enim eleifend felis pretium feugiat. Vivamus quis mi. Phasellus a est. Phasellus magna. In hac habitasse platea dictumst. Curabitur at lacus ac velit ornare lobortis. Curabitur a felis in nunc fringilla tristique. 

