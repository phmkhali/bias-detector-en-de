\chapter{Introduction}
Machine Translation (MT) is a sub-field of computational linguistics that uses software to translate text between languages \citep{linMachineTranslationAcademic2009}. It is part of Natural Language Processing (NLP), which belongs to the broader field of Artificial Intelligence (AI) \citep{smacchiaDoesAIReflect2024}. MT helps millions of people communicate across languages, in daily life and in areas like healthcare, law, and business \citep{kapplAreAllSpanish2025}. Services like Google Translate handle over 200 million users every day \citep{pratesAssessingGenderBias2019,shresthaExploringGenderBiases2022}.

The MT market is growing fast. A report by \citet{skyquestMachineTranslationMT2025} valued it at 980 million USD in 2023, with projections reaching 2.78 billion USD. New and more advanced translation models keep appearing, and many of them are free to use. As a result, MT tools are now used to translate large volumes of content across domains.

With this widespread use, the output of MT systems increasingly shapes how people receive and interpret information. But automatic translations are not neutral. There is growing concern about the social effects of biased translations. One key issue is gender bias. MT systems are often trained on large datasets that reflect social norms and stereotypes. If the data contains gender bias, the system will likely reproduce it \citep{choMeasuringGenderBias2019,soundararajanInvestigatingGenderBias2024,smacchiaDoesAIReflect2024}.

A common case is the use of gendered terms in translations of gender-neutral input. For example, the English sentence “The nurse is hard-working” does not say anything about gender. But a translation system may render it in German as “Die Krankenschwester ist fleißig,” which uses the explicitly feminine term Krankenschwester. Similarly, “The surgeon is hard-working” may become “Der Chirurg ist fleißig,” using the masculine form Chirurg. These choices add gendered assumptions that were not present in the original. Such patterns are not just technical side effects. They can reinforce stereotypes, especially when they appear in job ads, reports, or other public texts.


\section{Motivation}

\subsection{Social and Ethical Importance of Addressing Gender Bias}\label{section:social_and_ethical_importance_of_addressing}
Academia has come to the consensus that MT systems do default to male pronouns when gender in the source sentence is ambiguous. In addition, as shown in the earlier example where "the surgeon" and "the nurse" were translated with stereotypical genders, the reinforcement of occupational stereotypes is an increasing concern. When MT is used for job descriptions, recommendation letters, or resumes and it inserts or reinforces unfairly gendered language \citep{bolukbasiManComputerProgrammer2016}, it may discourage individuals whose gender is misrepresented or stereotyped. In turn, this would reduce their chances of success in recruitment processes. Failing to address this issue can bring broader consequences for business, leading to the exclusion of qualified candidates, reduce diversity and contradict international standards. Organizations like the United Nations, UNESCO, and the European Union stress the importance of gender equality and inclusive language, making gender equality one of the 17 Sustainable Development Goals for 2030 \citep{sczesnyCanGenderFairLanguage2016,unitednationsAchieveGenderEquality2023}. Moreover, language influences how people think. There have been consistent findings that speakers do not understand masculine forms as referring to both genders equally, but interpret them in a male based way \citep{sczesnyCanGenderFairLanguage2016}. Gender-fair language is more commonly used in official texts, muss less in private messages. Still, exposure matters. The more often people see inclusive forms, the more normal they become. Promoting awareness of these patterns is an important step toward reducing bias in society.

\subsection{Why Detection Systems Are Needed}

Current research on this topic tends to focus more on the quantitative measurement of gender bias \citep{rescignoGenderBiasMachine2023,barclayInvestigatingMarkersDrivers2024a,smacchiaDoesAIReflect2024}, e.g., counting the occurences of gendered pronouns or grammatical forms in outputs when prompting models with a neutral input. It is then often compared against a standard or desired outcome like real-world demographic distributions \citep{smacchiaDoesAIReflect2024,pratesAssessingGenderBias2019} or human evaluation \citep{lardelliBuildingBridgesDataset2024,savoldiWhatHarmQuantifying2024}. However, at this stage, evaluations are not enough for accountability. Few approaches address an active gender bias detection layer. While this gap remains in translation systems, similar issues have been addressed in other domains. For example, as summarized by \citet{shresthaExploringGenderBiases2022}, \citet{schwemmerDiagnosingGenderBias2020} propose a detection framework to uncover gender bias in facial recognition technologies. Their findings show that these systems are more accurate in identifying individuals as women when the images conform to stereotypical feminine features like long hair or makeup. In some cases, systems even associated such images with stereotypically gendered labels like "kitchen" or "cake," despite these elements not being present. 
A detection system specifically for MT would increase linguistic transparency, because without the development of bias-aware tools, problematic translations are likely to scale without oversight. Therefore, addressing gender bias in MT becomes both a social and ethical necessity.

\section{Problem Statement and Research Questions}
\textbf{DRAFT NEED TO REWRITE AFTER IMPLEMENTATION}
The core problem boils down to the significant bias towards the masculine form in English-German MTs, sometimes consituting 93-96\% of translations for isolated words \citep{lardelliBuildingBridgesDataset2024}. These outputs often reflect social stereotypes rather than objective translations, yet current systems offer no mechanism to detect or signal when such bias occurs \citep{rescignoGenderBiasMachine2023}. To address this, this thesis deploys a blackbox approach to explore how fine-tuning a pre-trained multilingual BERT model can help detect gender bias in MT outputs. The model takes an input sentence and its corresponding German translation and predicts whether the translation introduces gender bias. It focuses on identifying two common cases: added gendered pronouns and wrongly gendered nouns.

\textbf{DRAFT NEED TO REWRITE AFTER IMPLEMENTATION}
The translation system used is \href{https://github.com/Helsinki-NLP/Opus-MT?tab=readme-ov-file}{Opus-MT}, an open-source neural MT model. %mention which bert i will use% 
Translations are passed through BERT, trained on a dataset I have constructed by combining and adapting several existing datasets from other researchers. The classifier is lightweight and efficient, aiming for transparent behavior and easy integration into other tools \citep{devlinBERTPretrainingDeep2019}. Its predictions are used to highlight biased parts in a web-based demo. The goal is not to build a perfect detector, but a working proof of concept that shows how bias can be flagged automatically. This supports more critical use of MT systems and encourages further development of bias-aware translation tools.

The main research question is: \textbf{"How can a NLP-based binary classification model detect gender bias in English-German translations?"}. This involves building a suitable training dataset, selecting features that capture bias patterns, and evaluating how well the model generalizes across different domains.

\section{Scope}

\textbf{WRITE AFTER IMPLEMENTATION PART}
This thesis focuses only on English-to-German (EN-DE) MT. This language pair is widely used in MT reserach, offering high-quality models and datasets. Extending the work to other language pairs would require native-level understanding to detect subtle gender patterns and translation errors, which is outside the current scope. 

\section{Limitations}
\textbf{WRITE AFTER IMPLEMENTATION PART}
It becomes especially difficult to detect when sentences contain multiple subjects, indirect references, or ambiguous pronouns. For example, as \citet{barclayInvestigatingMarkersDrivers2024a} explain, the sentence “He went to see her mother” clearly implies three people, while “He went to see his mother” could refer to either two or three. These types of structures introduce ambiguity that makes annotation and evaluation much harder. Creating a dataset that captures such linguistic complexity would require significant effort and careful control of variables. One broader limitation in building datasets for complex scenarios with multiple subjects is the difficulty of isolating the influence of each gendered entity \citep{lardelliBuildingBridgesDataset2024}. When working with natural language sources, it becomes hard to tell what caused the bias in the translation. Because of this, the focus of this thesis is on simpler sentence structures with a single subject. This makes it easier to identify and explain bias patterns. It also fits the intended use case: translating business texts like job advertisements or reports, which rarely involve multiple nested clauses or ambiguous pronouns.
 

\section{Overview of Chapters}
\textbf{DRAFT NEED TO REWRITE AFTER IMPLEMENTATION}
