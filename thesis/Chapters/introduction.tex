\chapter{Introduction}
Machine Translation (MT) is a sub-field of computational linguistic that uses computer software to translate texts between languages \citep{linMachineTranslationAcademic2009}. It is a major area within Natural Language Processing (NLP), a branch of Artificial Intelligence (AI) \citep{smacchiaDoesAIReflect2024}. Millions of people rely on MT systems daily to overcome language barriers in various domains such as medicine, business, diplomacy and everyday life \citep{kapplAreAllSpanish2025}. Tools like Google Translate serve over 200 million users daily \citep{pratesAssessingGenderBias2019,shresthaExploringGenderBiases2022}, with new advanced translation models appearing on the market frequently. According to a market analysis by \citet{skyquestMachineTranslationMT2025}, the MT market size was valued at 980 million USD in 2023 and is projected to reach 2.78 billion USD by 2023. 

With this growing availability and accessibility of free MT tools capable of handling complex sentences, their use in translating large volumes of online content is increasing \citep{thompsonShockingAmountWeb2024}. This not only expands their influence on global access to information, but also shapes how readers perceive and interpret that content. Automated and unsupervised translations raise new concerns: not just about quality, but also about bias. One aspect is gender bias. Several studies \citep{smacchiaDoesAIReflect2024,choMeasuringGenderBias2019,stanczakSurveyGenderBias2021,soundararajanInvestigatingGenderBias2024} confirm that MT systems trained on large-scale datasets that incorporate societal biases, can learn and perpetuate gender biases present in the training data. This includes, but is not limited to, associating occupations with a genders, inserting gender-specific pronouns for gender-neutral terms, and similar assumptions.

There is a risk of incorrect gender assignment when translating between languages with and without grammatical gender. For example, the gender-neutral English sentences "The surgeon is hard-working" and "The nurse is hard-working" are translated into German as "Der Chirurg ist fleißig" and "Die Krankenschwester ist fleißig" respectively, as seen in \autoref{fig:gt_surgeon_example} and \autoref{fig:deepL_surgeon_example}. These translations introduce occupational gender stereotypes. “Der Chirurg” is the masculine form of “surgeon” in German, adding a male gender where the original English sentence did not specify one. Similarly, “Die Krankenschwester” is the feminine form of “nurse,” again assigning a gender that was not present in the source. These patterns are not just technical flaws. They can reinforce harmful stereotypes in real-world contexts. The following section outlines the broader motivation behind this thesis.


\section{Motivation}

Academia has come to the consensus that MT systems do default to male pronouns when gender in the source sentence is ambiguous. In addition, as shown in the earlier example where "the surgeon" and "the nurse" were translated with stereotypical genders, the reinforcement of occupational stereotypes is an increasing concern as well. When MT is used for job descriptions, recommendation letters, or resumes and it inserts or reinforces unfairly gendered language \citep{bolukbasiManComputerProgrammer2016}, it may discourage individuals whose gender is misrepresented or stereotyped. In turn, this would reduce their chances of success in recruitment processes. 
This may also bring broader consequences for businesses. Failing to address this issue can lead to the exclusion of qualified candidates, reduce diversity and contradict international standards. Organizations like the United Nations, UNESCO, and the European Union stress the importance of gender equality and inclusive language \citep{sczesnyCanGenderFairLanguage2016,unitednationsAchieveGenderEquality2023}.
Ethical AI guidelines from global institutions also highlight the need for fair outcomes \citep{ullmannGenderBiasMachine2022}, so businesses need to meet these standards if they want to stay credible and act responsibly. 

Current research on this topic tends to focus more on the quantitative measurement of gender bias \citep{rescignoGenderBiasMachine2023,barclayInvestigatingMarkersDrivers2024a,smacchiaDoesAIReflect2024}, e.g. counting the occurences of gendered pronouns or grammatical forms in outputs when prompting models with a neutral input. It is then often compared against a standard or desired outcome like real-world demographic distributions \citep{smacchiaDoesAIReflect2024,pratesAssessingGenderBias2019} or human evaluation \citep{lardelliBuildingBridgesDataset2024,savoldiWhatHarmQuantifying2024}. However, current evaluations are not enough for accountability. Few approaches address an active gender bias detection layer. While this gap remains in translation systems, similar issues have been addressed in other domains. For example, as summarized by \citet{shresthaExploringGenderBiases2022}, \citet{schwemmerDiagnosingGenderBias2020} propose a detection framework to uncover gender bias in facial recognition technologies. Their findings show that these systems are more accurate in identifying individuals as women when the images conform to stereotypical feminine features like long hair or makeup. In some cases, systems even associated such images with stereotypically gendered labels like "kitchen" or "cake," despite these elements not being present. 
A detection system specifically for MT would increase linguistic transparency and make room for new findings, because without the development of bias-aware tools, problematic translations are likely to scale without oversight. Therefore, addressing gender bias in MT becomes both a social and ethical necessity.




\section{Problem Statement and Research Questions}

Vestibulum fringilla pede sit amet augue. In turpis. Pellentesque posuere. Praesent turpis. Aenean posuere, tortor sed cursus feugiat, nunc augue blandit nunc, eu sollicitudin urna dolor sagittis lacus. Donec elit libero, sodales nec, volutpat a, suscipit non, turpis. Nullam sagittis. Suspendisse pulvinar, augue ac venenatis condimentum, sem libero volutpat nibh, nec pellentesque velit pede quis nunc. Vestibulum ante ipsum primis in faucibus orci luctus et ultrices posuere cubilia Curae; Fusce id purus. Ut varius tincidunt libero. Phasellus dolor. Maecenas vestibulum mollis diam. Pellentesque ut neque. Pellentesque habitant morbi tristique senectus et netus et malesuada fames ac turpis egestas. In dui magna, posuere eget, vestibulum et, tempor auctor, justo. In ac felis quis tortor malesuada pretium. Pellentesque auctor neque nec urna. Proin sapien ipsum, porta a, auctor quis, euismod ut, mi. Aenean viverra rhoncus pede. Pellentesque habitant morbi tristique senectus et netus et malesuada fames ac turpis egestas. Ut non enim eleifend felis pretium feugiat. Vivamus quis mi. Phasellus a est. Phasellus magna. In hac habitasse platea dictumst. Curabitur at lacus ac velit ornare lobortis. Curabitur a felis in nunc fringilla tristique.

\section{Aufbau der Arbeit}

Vestibulum fringilla pede sit amet augue. In turpis. Pellentesque posuere. Praesent turpis. Aenean posuere, tortor sed cursus feugiat, nunc augue blandit nunc, eu sollicitudin urna dolor sagittis lacus. Donec elit libero, sodales nec, volutpat a, suscipit non, turpis. Nullam sagittis. Suspendisse pulvinar, augue ac venenatis condimentum, sem libero volutpat nibh, nec pellentesque velit pede quis nunc. Vestibulum ante ipsum primis in faucibus orci luctus et ultrices posuere cubilia Curae; Fusce id purus. Ut varius tincidunt libero. Phasellus dolor. Maecenas vestibulum mollis diam. Pellentesque ut neque. Pellentesque habitant morbi tristique senectus et netus et malesuada fames ac turpis egestas. In dui magna, posuere eget, vestibulum et, tempor auctor, justo. In ac felis quis tortor malesuada pretium. Pellentesque auctor neque nec urna. Proin sapien ipsum, porta a, auctor quis, euismod ut, mi. Aenean viverra rhoncus pede. Pellentesque habitant morbi tristique senectus et netus et malesuada fames ac turpis egestas. Ut non enim eleifend felis pretium feugiat. Vivamus quis mi. Phasellus a est. Phasellus magna. In hac habitasse platea dictumst. Curabitur at lacus ac velit ornare lobortis. Curabitur a felis in nunc fringilla tristique. 

