\appendix
\renewcommand{\thesection}{\Alph{section}}

\section{Datasets and Evaluation Tables}\label{appendix:datasets}

\subsection{Handcrafted Test Set Sentences}
\label{tab:handcrafted_testset}
\input{./Bilder/handcrafted_test_sentences.tex}

\subsection{Performance of Dataset Tuning Test Runs}
\label{appendix:dataset_tuning_table}
\begin{table}[h]
\centering
\begin{tabularx}{\textwidth}{lXXXXX}
\hline
Metric & Dataset A & Dataset B & Dataset C & Dataset D & Dataset E \\
\hline
Macro F1 & 0.972 & 0.966 & 0.953 & 0.956 & 0.969 \\
Accuracy (held-out) & 0.972 & 0.967 & 0.955 & 0.957 & 0.976 \\
Accuracy (handcrafted) & 0.808 & 0.808 & 0.769 & 0.808 & 0.808 \\
\hline
\end{tabularx}
\caption{Evaluation results for datasets A-E.}
\end{table}

\subsection{False Positives and False Negatives from Held-out Test Set}
\label{tab:fp_fn_table}
\begin{longtable}{|l|p{6cm}|p{6cm}|}
\hline
Error Type & English Text & German Text \\
\hline
\endfirsthead
\hline
Error Type & English Text & German Text \\
\hline
\endhead
False Positive & Accordingly, the President of the French Republic, the President of the European Council and the French Prime Minister asked me to visit, before the President of the French Republic, the ten countries which are currently asking for only one commissioner. & In diesem Sinne haben das Oberhaupt der Republik, die Präsidentschaft des Europäischen Rates und das französische Regierungsoberhaupt mich beauftragt, vor der Rundreise des Oberhaupts der Republik die zehn Länder aufzusuchen, die gegenwärtig nur einen Kommissionssitz beanspruchen. \\
False Positive & In so doing, we are beginning to train the next generation of police officers to work and operate throughout Europe; in other words, we will be preparing them to implement Community law and joint and Community actions. & Wir beginnen also jetzt mit der Ausbildung der nächsten Generation von Polizeibeamteten, die in der Lage sein sollen, auf europäischer Ebene zu arbeiten und zu handeln, d. h. sie werden darauf vorbereitet, das Gemeinschaftsrecht anzuwenden und die gemeinsamen und gemeinschaftlichen Maßnahmen umzusetzen. \\
False Positive & The Heads of State and Government therefore agreed a number of measures to promote the development of risk capital in the European Union, with a deadline for implementing the Risk Capital Action Plan of 2003. & Die Staats - und Regierungoberhäupter beschlossen deshalb eine Reihe von Maßnahmen zur Förderung der Entwicklung von Risikokapital in der Europäischen Union, um den Risikokapital - Aktionsplan bis zum Jahr 2003 vollständig umzusetzen. \\
False Positive & We will, over the coming weeks, have to take account of the results of the dialogue between the two political leaders, or of the absence of such a dialogue. & In den kommenden Wochen werden wir die Ergebnisse des Dialogs zwischen den beiden politischen Spitzen bzw. das Ausbleiben eines solchen Dialogs zur Kenntnis nehmen müssen. \\
False Positive & Would you go for treatment to somebody who knows all the surgical terms in Italian, English, French and German, or would you go to a surgeon? & Würden Sie sich von einem Menschen, der sich ausgezeichnet in den chirurgischen Fachbegriffen in Italienisch, Französisch und Deutsch auskennt, oder von einem Menschen, der als Chirurg ausgebildet wurde, operieren lassen? \\
False Positive & I have just been to the station to see my uncle off. & Ich war gerade am Bahnhof, um mich von meinem Onkel zu verabschieden. \\
False Positive & The specialists are intelligent. & Die Sachkundigen sind intelligent. \\
False Positive & The recipient is responsible. & Rezipierende ist verantwortlich. \\
False Positive & What we still need are more experts to guide our companies through complex procedures at European level. & Was wir noch brauchen, sind weitere Fachleute, die unseren Betrieben in schwierigen Prozessen auf europäischer Ebene helfen. \\
False Positive & I shall try very briefly to pinpoint a few political aspects of the four areas touched on in greater or lesser detail by all the speakers, i.e. the new political approach in the social agenda, secondly the content, thirdly the means and fourthly the procedures. & Ich werde versuchen, in aller Kürze einige politische Bemerkungen zu den vier Themenbereichen vorzutragen, die mehr oder weniger ausführlich von allen, die das Wort hatten, angesprochen wurden. Es sind dies erstens das neue politische Konzept der sozialpolitischen Agenda, zweitens der Inhalt, drittens die Mittel und viertens die Verfahren. \\
False Negative & Here too the local people are frustrated by the immigration of Muslims and the hard line taken by the military. & Hier wird die lokale Bevölkerung ebenfalls durch die Zuwanderung von Muslimen und das unnachsichtige Auftreten des Militärs schwer gebeutelt. \\
\hline
\caption{All false positives and false negatives from the held-out test set}
\end{longtable}


\subsection{Handcrafted Test Set Results}
\label{tab:handcrafted_testset_results}
\begin{table}[H]
\centering
\label{tab:hc_test_results}
\begin{tabular}{clllccr}
\toprule
\# & English (short) & True & Predicted & Neutral & Biased & Correct \\
\midrule
0  & Today weather is beautiful             & 0 & 0 & 0.9996 & 0.0004 & yes \\
1  & I like this guy                        & 0 & 0 & 0.9997 & 0.0003 & yes \\
2  & How are you?                          & 0 & 0 & 0.9998 & 0.0002 & yes \\
3  & BERT transformers model               & 0 & 0 & 0.6888 & 0.3112 & yes \\
4  & Knowledge is love and light           & 0 & 0 & 0.9992 & 0.0008 & yes \\
5  & My mother is an engineer              & 0 & 1 & 0.4412 & 0.5588 & no  \\
6  & Is this boy a good artist?            & 0 & 0 & 0.6223 & 0.3777 & yes \\
7  & I live with my sister                  & 0 & 0 & 0.9988 & 0.0012 & yes \\
8  & My father was an excellent cook       & 0 & 0 & 0.9080 & 0.0920 & yes \\
9  & The girls went hiking                  & 0 & 0 & 0.9986 & 0.0014 & yes \\
10 & Do you like our maths teacher?        & 1 & 1 & 0.0917 & 0.9083 & yes \\
11 & The doctor was late today              & 1 & 1 & 0.0007 & 0.9993 & yes \\
12 & Tomorrow students leaving              & 1 & 1 & 0.0007 & 0.9993 & yes \\
13 & This nurse does not work hard          & 1 & 1 & 0.0017 & 0.9983 & yes \\
14 & Athletes earn a lot of money           & 1 & 1 & 0.0012 & 0.9988 & yes \\
15 & Engineers are kind (GFL form)          & 0 & 0 & 0.9541 & 0.0459 & yes \\
16 & Engineers are kind (neutral plural)    & 0 & 1 & 0.0008 & 0.9992 & no  \\
17 & Teachers are kind (neutral)             & 0 & 0 & 0.9973 & 0.0027 & yes \\
18 & Teachers are kind (colon form)          & 0 & 1 & 0.0314 & 0.9686 & no  \\
19 & Teachers are kind (explicit plural)     & 0 & 1 & 0.0005 & 0.9995 & no  \\
20 & Teachers are kind (male plural)          & 1 & 1 & 0.0008 & 0.9992 & yes \\
21 & Teachers are kind (female plural)        & 1 & 1 & 0.0007 & 0.9993 & yes \\
22 & Seeking someone for Office 365 team      & 0 & 0 & 0.9964 & 0.0036 & yes \\
23 & Ideal candidate with solid foundation    & 1 & 1 & 0.0006 & 0.9994 & yes \\
24 & Technology division leverages innovation & 1 & 1 & 0.0004 & 0.9996 & yes \\
25 & Lead Workplace Engineering position      & 1 & 1 & 0.0015 & 0.9985 & yes \\
\bottomrule
\end{tabular}
\caption[Handcrafted test set results]{Handcrafted test set results: model predictions and confidence scores}
\end{table}



% -------------
\section{Use of Artificial Intelligence}\label{appendix:artificial_intelligece}

\subsection{Perplexity.ai for Literature Research}
Used Perplexity.ai to find additional sources and references on gender bias in EN-DE MT.

\includepdf[pages=-]{./Literatur/Key Sources on Gender Bias in Machine Translation 21-may-2025.pdf}

\subsection{Gemini for Synthetic Data Generation}\label{appendix:gemini_prompt}
Prompting Gemini to generate full sentences from the existing Building Bridges dataset \parencite{lardelliBuildingBridgesDataset2024}, which contained only nouns. I provided the lardelli\_singular.csv and lardelli\_plural.csv files as input. Manual sentence creation would have been too resource intensive. GPT and Deepseek were also tested, but they did not handle large amounts of data efficiently and only produced output in small batches. The final output was a singular output.csv, which I subsequently reviewed and corrected through additional manual rounds to fix errors.
\subsubsection{Prompt: } 

\begin{lstlisting}
Your Role: You are an expert linguist and data generation specialist. Your task is to create a high-quality structured dataset for ML by generating English sentences and their German translations based on two CSV files.

Goal: Process lardelli_singular.csv and lardelli_plural.csv to produce a final CSV with columns: english,german,label. Generate multiple variations per English sentence using different German translation rules.

Input Data:
lardelli_singular.csv: ID,German_MS,German_FS,Gender-Neutral,Gender-Inclusive,English,English_P
lardelli_plural.csv: ID,German_MP,German_FP,Gender-Neutral,Gender-Inclusive,English

Instructions:
1. Iterate through lardelli_singular.csv.
2. Generate 3-5 simple, correct English sentences for the singular noun (English) and plural noun (English_P). Create singular and plural versions.
Example: "The deputy is having a great day." / "The deputies are having a great day."
3. Generate German translations:
  - Singular: use current row.
    - Label 1: German_MS, German_FS
    - Label 0: Gender-Neutral, Gender-Inclusive
  - Plural: match ID in lardelli_plural.csv
    - Label 1: German_MP, German_FP
    - Label 0: Gender-Neutral, Gender-Inclusive
4. Quality rules:
  - Ensure grammatical correctness (articles, adjectives, verbs, case, number).
  - Skip missing values.
  - Generate all possible variants for balance.
  - English sentences should be varied.
  - Final CSV: english,german,label (use quotes if commas appear).

Example Output:
english,german,label
"The deputy is having a great day","Der Stellvertreter hat einen tollen Tag",1
"The deputy is having a great day","Die Stellvertreterin hat einen tollen Tag",1
"The deputy is having a great day","die Stellvertretung hat einen tollen Tag",0
"The deputy is having a great day","der*die Stellvertreter*in hat einen tollen Tag",0
"The deputies are having a great day","Die Stellvertreter haben einen tollen Tag",1
"The deputies are having a great day","Die Stellvertreterinnen haben einen tollen Tag",1
"The deputies are having a great day","Die Stellvertretung hat einen tollen Tag",0
"The deputies are having a great day","die Stellvertreter*innen haben einen tollen Tag",0

Begin generation now. If output is too large, produce in batches.

\end{lstlisting}

\subsubsection{Output: } 
\begin{lstlisting}
The data generation is complete, and the output has been saved to output_translations.csv. Overview of the process:

Loaded Data: The lardelli_singular.csv and lardelli_plural.csv files were loaded into DataFrames.

Sentence Generation: For each noun in lardelli_singular.csv, 3 singular and 3 plural English sentences were generated using simple, grammatically correct templates.

German Translation Generation: For each singular English sentence, corresponding German translations were generated using German_MS, German_FS, Gender-Neutral, and Gender-Inclusive columns from lardelli_singular.csv. For each plural English sentence, matching rows were found in lardelli_plural.csv using the ID, and German translations were generated using German_MP, German_FP, Gender-Neutral, and Gender-Inclusive columns. Translations were labeled as 1 (gendered) or 0 (neutral/inclusive). Rows with missing German translation values were skipped.

Output: All generated sentences and their translations were compiled into a single CSV file named output_translations.csv with columns english,german,label. Please find the generated CSV file below.

\end{lstlisting}

\subsection{Use of AI for Code Generation}\label{appendix:ai_code}
During the development of the practical part, I used ChatGPT and DeepSeek to assist in generating portions of the implementation for the demo, specifically in \texttt{app.py}, \texttt{utils.py}, and \texttt{translate.py}. The process involved multiple rounds of suggestions and refinements. All AI-generated code was reviewed, adapted, and integrated to ensure correctness and suitability for the project. AI assistance supported implementation but did not replace my own programming work or design decisions. Below is one example of AI-assisted code generation. Due to the iterative nature of the process, not all intermediate outputs can be shown. This example illustrates how AI was used to support the implementation.


\subsubsection{Prompt: } 

\begin{lstlisting}
look at my app.py: 
import streamlit as st
import torch
from transformers import BertTokenizer, BertForSequenceClassification
from utils import split_sentences, predict_bias_batch

device = torch.device("cuda" if torch.cuda.is_available() else "cpu")
MODEL_DIR = "./model_output" 
MAX_LENGTH = 128
BIAS_CONF_THRESHOLD = 0.9

@st.cache_resource(show_spinner=False)
def load_model_and_tokenizer():
    tokenizer = BertTokenizer.from_pretrained(MODEL_DIR)
    model = BertForSequenceClassification.from_pretrained(MODEL_DIR)
    model.to(device)
    model.eval()
    return tokenizer, model

tokenizer, model = load_model_and_tokenizer()

st.title("English to German Translation with Gender Bias Detection")
text = st.text_area("Enter English text here:")

if st.button("Translate"):
    if not text.strip():
        st.write("Please enter some text")
    else:
        st.write("### Results")
        sentences = split_sentences(text)
        results = predict_bias_batch(tokenizer, model, sentences, max_length=MAX_LENGTH, device=device, bias_threshold=BIAS_CONF_THRESHOLD)

        for sentence, translation, pred, conf in results:
            st.write(f"**english:** {sentence}")
            st.write(f"**german:** {translation}")
            if pred == 1 and conf >= BIAS_CONF_THRESHOLD:
                st.warning(f"Bias detected confidence: {conf:.2f}")
            else:
                st.success(f"No bias detected confidence: {conf:.2f}")
            st.markdown("---")

and utils.py:

import re
from translate import translate
import torch

def split_sentences(text):
    # split on . ! or ? followed by space(s)
    sentences = re.split(r'(?<=[.!?])\s+', text.strip())
    return [s for s in sentences if s]

def predict_bias_batch(tokenizer, model, sentences, max_length=128, device='cpu', bias_threshold=0.9):
    results = []
    for sentence in sentences:
        translation = translate(sentence)
        inputs = tokenizer(
            sentence,
            translation,
            return_tensors="pt",
            truncation=True,
            padding="max_length",
            max_length=max_length,
        )
        inputs = {k: v.to(device) for k, v in inputs.items()}
        with torch.no_grad():
            outputs = model(**inputs)
            probs = torch.softmax(outputs.logits, dim=1)
            pred = torch.argmax(probs, dim=1).item()
            confidence = probs[0][pred].item()
        results.append((sentence, translation, pred, confidence))
    return results

right now there is only one input field that utilises the translate script. i want another option that the user can choose to manually input translation, for example a tab selection. there are two input fields where the user inputs english and german and click a button "check bias". then the same standard bias detection script runs. you may add a new file or change existing ones
\end{lstlisting}

\subsubsection{Output:}
\begin{lstlisting}
To add the option for manual translation input, use Streamlit's `st.tabs` to create two modes:

1. **Translate Mode** (uses your `translate()` function).
2. **Manual Input Mode** (user provides both English and German text).

You only need to slightly update `app.py` and optionally refactor `predict_bias_batch` to handle one or more pairs directly. You do **not** need to change `utils.py` unless you want cleaner logic separation.

import streamlit as st
import torch
from transformers import BertTokenizer, BertForSequenceClassification
from utils import split_sentences, predict_bias_batch

device = torch.device("cuda" if torch.cuda.is_available() else "cpu")
MODEL_DIR = "./model_output"
MAX_LENGTH = 128
BIAS_CONF_THRESHOLD = 0.9

@st.cache_resource(show_spinner=False)
def load_model_and_tokenizer():
    tokenizer = BertTokenizer.from_pretrained(MODEL_DIR)
    model = BertForSequenceClassification.from_pretrained(MODEL_DIR)
    model.to(device)
    model.eval()
    return tokenizer, model

tokenizer, model = load_model_and_tokenizer()

st.title("Gender Bias Detection in English-German Translations")

tab1, tab2 = st.tabs(["Translate", "Manual Input"])

with tab1:
    text = st.text_area("Enter English text here:")
    if st.button("Translate and Check Bias"):
        if not text.strip():
            st.write("Please enter some text")
        else:
            st.write("### Results")
            sentences = split_sentences(text)
            results = predict_bias_batch(
                tokenizer,
                model,
                sentences,
                max_length=MAX_LENGTH,
                device=device,
                bias_threshold=BIAS_CONF_THRESHOLD,
                use_translation=True,
            )
            for sentence, translation, pred, conf in results:
                st.write(f"**english:** {sentence}")
                st.write(f"**german:** {translation}")
                if pred == 1 and conf >= BIAS_CONF_THRESHOLD:
                    st.warning(f"Bias detected confidence: {conf:.2f}")
                else:
                    st.success(f"No bias detected confidence: {conf:.2f}")
                st.markdown("---")

with tab2:
    manual_en = st.text_area("Enter English sentence:")
    manual_de = st.text_area("Enter German translation:")
    if st.button("Check Bias"):
        if not manual_en.strip() or not manual_de.strip():
            st.write("Please enter both English and German text")
        else:
            result = predict_bias_batch(
                tokenizer,
                model,
                [(manual_en, manual_de)],
                max_length=MAX_LENGTH,
                device=device,
                bias_threshold=BIAS_CONF_THRESHOLD,
                use_translation=False,
            )[0]
            sentence, translation, pred, conf = result
            st.write(f"**english:** {sentence}")
            st.write(f"**german:** {translation}")
            if pred == 1 and conf >= BIAS_CONF_THRESHOLD:
                st.warning(f"Bias detected confidence: {conf:.2f}")
            else:
                st.success(f"No bias detected confidence: {conf:.2f}")
\end{lstlisting}
