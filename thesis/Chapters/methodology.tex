\chapter{Methodology}

\section{Workflow}
The project starts by selecting and combining datasets from prior work. The data is cleaned and pre-processed to prepare it for feature extraction. Features are then created to represent the data for model training. The model is chosen and trained with tuned hyperparameters. Its performance is evaluated using specific metrics. Finally, a demo application integrates the model and data, allowing interaction and testing.

\section{Dataset Handling}
% thought process for selecting datasets from prior works
% experimental joining and testing of datasets
\subsection{Final Dataset}
% description of the final dataset creation
% composition and characteristics

\section{Data Pre-processing}
% how data is cleaned and prepared before feature extraction
% tokenization, normalization, etc.

\section{Feature extraction}
% how features are extracted from the data
% what features are used

\section{Model Selection and Training}
% choice of model architecture and reasoning
% training approach overview
\subsection{Hyperparameters}
% experimentation with epochs and other key parameters

\section{Evaluation}
% metrics and evaluation strategy
% validation approach

\section{Demo Application Design}
% concept and structure of the demo app
% how it connects with other components