\chapter{Methodology}
This chapter explains the overall approach and structure of the project. It covers how data is handled, how the model is built and trained, and how the demo application is designed. The goal is to present the key methodological steps and rationale guiding the system’s development.

\section{Workflow}
The project begins by selecting and combining datasets from previous work (see \autoref{fig:workflow}). The model building phase then follows, as shown in the purple boxes. It starts with cleaning and preparing the data, followed by extracting features for training. A pre-trained German BERT model is then fine-tuned for the classification task. Its performance is measured using standard evaluation metrics. In the final step, the trained model is integrated into a demo application for user interaction and testing.

\vspace{1cm} 
\begin{figure}[ht]
    \centering
    \scalebox{0.8}{\tikzstyle{startstop} = [rectangle, rounded corners, minimum width=3cm, minimum height=1cm,text centered, draw=black, fill=gray!20]
\tikzstyle{process} = [rectangle, minimum width=3cm, minimum height=1cm, text centered, draw=black, fill=blue!10]
\tikzstyle{arrow} = [thick,->,>=stealth]

\begin{tikzpicture}[node distance=1.7cm]

\node (start) [startstop] {Select and Combine Datasets};
\node (clean) [process, below of=start] {Clean and Pre-process Data};
\node (features) [process, below of=clean] {Extract Features};
\node (train) [process, below of=features] {Train Model};
\node (evaluate) [process, below of=train] {Evaluate Model Performance};
\node (demo) [startstop, below of=evaluate] {Show Model in Demo Application};

\draw [arrow] (start) -- (clean);
\draw [arrow] (clean) -- (features);
\draw [arrow] (features) -- (train);
\draw [arrow] (train) -- (evaluate);
\draw [arrow] (evaluate) -- (demo);

\end{tikzpicture}
}
    \caption{Workflow of the project.}
    \label{fig:workflow}
\end{figure}
\vspace{1cm} 

\section{Dataset Handling}
% thought process for selecting datasets from prior works
% experimental joining and testing of datasets

Since there was no ready-to-use dataset for this task and no prior work that built a similar model, I first had to define: (a) the desired structure and content of my dataset, and (b) the number of samples required.

Based on the three groupings I defined in \autoref{subsection:manifestations_of_gb}, 

\subsection{Final Dataset}
% description of the final dataset creation
% composition and characteristics

\section{Data Pre-processing}
% how data is cleaned and prepared before feature extraction
% tokenization, normalization, etc.

\section{Feature extraction}
% how features are extracted from the data
% what features are used

\section{Model Selection and Training}
% choice of model architecture and reasoning
% training approach overview
\subsection{Hyperparameters}
% experimentation with epochs and other key parameters

\section{Evaluation}
% metrics and evaluation strategy
% validation approach

\section{Demo Application Design}
% concept and structure of the demo app
% how it connects with other components