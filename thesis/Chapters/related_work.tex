\chapter{Theoretical Background and Related Work}
This section outlines key findings of related work on gender bias in MT, with a focus on the English-German (EN-DE) language pair to build the theoretical knowledge base. The research aims are to (1) define the core concept of gender bias in MT, (2) establish the relevance of the topic, (3) identify the research gap, and (4) justify technical design choices. 

For the literature review I combined incremental and conceptual literature review methods, where each source led to the identification of the next. Based on this progression, I identified key concepts and used them to organize and interpret the literature, aligning with a conceptual approach. The structure followed the qualitative Information Systems framework by \citet{schryenWritingQualitativeLiterature2015} and was further informed by \citet{shresthaExploringGenderBiases2022} and \citet{savoldiDecadeGenderBias2025}, who both conducted systematic reviews on gender bias in ML and MT respectively. 

% --------------------------------------------------------------------------------
\section{Literature Search Process}

\subsection{Search Sources and Tools}
Sources were primarily searched on \href{https://scholar.google.com/}{Google Scholar} and \href{https://www.perplexity.ai/}{Perplexity}, which served as an additional search engine. Prompts and outputs from Perplexity have been saved and are included in the appendix. To organize and manage the collected sources, \href{https://www.zotero.org/}{Zotero} was used throughout the process.

\subsection{Literature Review Framing}

To answer the four research aims, I have defined the key concepts in \autoref{tab:key-concepts}. Key search terms consisted of \textit{gender bias}, \textit{machine translation}, \textit{AI}, \textit{machine learning}, \textit{German}, \textit{stereotypes}, and \textit{detection}, which were combined with \textit{AND/OR}. The focus was on literature published between 2019 and 2025 to maintain relevance and currency, while foundational and definitional works from earlier periods were selectively included. The initial search for the term \textit{gender bias in machine translation} returned over 18,000 results. Through my iterative selection process, this was narrowed down to 34 core sources.

\renewcommand{\arraystretch}{1.3}
\begin{table}[ht!]
\centering
\begin{tabularx}{\textwidth}{>{\raggedright\arraybackslash}p{6.5cm}X}
\toprule
\textbf{Key Concept} & \textbf{Description} \\
\midrule

Foundations of Gender Bias in Natural Language Processing & Traces early research that identified gender bias in language. Focuses on foundational studies that showed why the issue matters and how later work builds on these findings. \\

Sources and Manifestations of Bias & Explains how stereotypes shape language and persist over time. Describes how societal bias enters training data, model design, and system feedback. Shows how bias appears in machine translation and everyday language. \\

Linguistic Challenges in English-German Translations & Explores key grammatical differences between English and German that affect translation. Focuses on how the lack of gender in English and its presence in German can lead to biased outputs. \\

Mitigation Strategies and Current Limitations & Reviews how current research tries to reduce gender bias in NLP. Highlights what these methods can and cannot do. Helps identify where a classification-based approach could fill gaps and improve bias detection in translations. \\

\bottomrule
\end{tabularx}
\caption{Key concepts relevant to this thesis}
\label{tab:key-concepts}
\end{table}


\subsection{Citation Tracking}
Backward citation searching involved reviewing references cited by selected papers, prioritizing frequently cited and foundational works relevant to gender bias in MT. Forward citation searching used Google Scholar’s "cited by" function to identify newer research citing those key papers. Filtering with specific terms (e.g., \textit{German} and \textit{machine translation}) was applied during forward search to maintain focus. Beyond these systematic methods, I also included supplementary sources when needed while writing. These consist of contextual references, statistics, or secondary citations that support specific points but were not part of the core conceptual or methodological framework. Supplementary sources were defined as materials identified outside the systematic search, such as papers found through backward citations or targeted queries for statistics and news, which provided support for subordinate arguments without being central to the study’s theoretical or analytical structure.



\subsection{Selection Criteria and Screening Process}\label{subsection:selection_criteria}
Titles and abstracts were manually screened to select relevant studies. \textbf{Inclusion criteria} required sources to specifically address gender bias in MT, provide examples or discussions of gender-related errors, or explain the significance of gender bias in this context. Sources also had to be available in full text without access restrictions. \textbf{Exclusion criteria} filtered out studies focusing on general NLP bias without a direct link to MT, non-gender biases, and highly technical papers lacking contribution to the general understanding of gender bias or that did not provide additional knowledge beyond what was already found in previously published papers. Full texts were reviewed after initial screening to confirm relevance and extract insights. Redundant sources not providing new perspectives aligned with the thesis goals were excluded.

% --------------------------------------------------------------------------------

\section{Understanding Gender Bias in English-to-German Machine Translation}

\subsection{Differentiating between Natural Language Processing and Machine Translation}
\textbf{Natural Language Processing (NLP)} refers to the development of machine systems that can process and generate human language. The goal is to mimic and understand it as fluently as possible \citep{smacchiaDoesAIReflect2024,ullmannGenderBiasMachine2022}. Common applications are chatbots, translation tools, speech recognition, and image captioning.

\textbf{MT} is a direct application of NLP. It is used to automatically translate text from one language to another \citep{linMachineTranslationAcademic2009}. MT systems have gone through several stages of development; earlier approaches like rule-based and statistical MT used manually defined grammar rules or pattern matching from large translation corpora \citep{chakravarthiSurveyOrthographicInformation2021}. For example:

\begin{quote}
"The girl reads a book" → "Das Mädchen liest ein Buch"\\[0.5em]
Rules: "girl" → "Mädchen", "reads" → "lesen", "book" → "Buch"
\end{quote}

\noindent These systems often struggled with full sentences and complex expressions because they fail to capture context and phrase-level meaning. "She gave him a hand" might be translated literally, missing its idiomatic meaning.

Most modern systems, including Google Translate and DeepL, use \textbf{neural machine translation (NMT)} \citep{wuGooglesNeuralMachine2016,deeplHowDoesDeepL2021}. These systems are trained on large sets of translated texts. They learn to represent the meaning of whole sentences as mathematical structures and generate more fluent and accurate translations. Unlike earlier systems, they aim to consider the full context of a sentence, which helps reduce errors and improves the handling of ambiguous or idiomatic language.

In short, MT has evolved from fixed rules to data-driven systems that handle context, but unlike large language models (LLMs), MT is trained specifically for translation tasks rather than general language understanding.

\subsection{What is Gender Bias in Machine Translation?}

% --------------------------------------------------------------------------------

\section{Societal Relevance and Impact of Gender Bias in Machine Translation}

% --------------------------------------------------------------------------------

\section{Research Gaps in Gender Bias Detection for English-to-German Machine Translation}

% --------------------------------------------------------------------------------

\section{Approach and Justification of the Technical Setup}

