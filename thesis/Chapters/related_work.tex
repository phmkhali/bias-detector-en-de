\chapter{Related Work}

This section outlines key findings of related work on gender bias in MT, with a focus on the English-German language pair. The research aims are to (1) define the core concept of gender bias in MT, (2) establish the relevance of the topic, (3) identify the research gap, and (4) justify technical design choices. To support this, I examine datasets, model types, and tools used in previous studies.

For the literature review I combined incremental and conceptual literature review methods, where each source led to the identification of the next. Based on this progression, I identified key concepts and used them to organize and interpret the literature, aligning with a conceptual approach. The structure followed the qualitative Information Systems framework by \citet{schryenWritingQualitativeLiterature2015} and further informed by \citet{shresthaExploringGenderBiases2022}, who conducted a systematic review on gender bias in ML and AI. These sources supported the development of search strategies, the selection of thematic areas, and the justification of the methodological approach.


\section{Literature Search Process}

\subsection{Search Sources and Tools}
Sources were primarily searched on \href{https://scholar.google.com/}{Google Scholar} and \href{https://www.perplexity.ai/}{Perplexity}, which served as an additional search engine. Prompts and outputs from Perplexity have been saved and are included in the appendix. To organize and manage the collected sources, \href{https://www.zotero.org/}{Zotero} was used throughout the process.

\subsection{Literature Review Framing}

To answer the four research aims, I have defined the key concepts in \autoref{tab:key-concepts}. Key search terms included \textit{gender bias}, \textit{machine translation}, \textit{AI}, \textit{machine learning}, \textit{German}, and \textit{stereotypes}. The focus was on literature published between 2019 and 2025 to maintain relevance and currency, while foundational and definitional works from earlier periods were selectively included.

\renewcommand{\arraystretch}{1.3}
\begin{table}[ht!]
\centering
\begin{tabularx}{\textwidth}{lX}
\toprule
\textbf{Key Concept} & \textbf{Description} \\
\midrule
Stereotypes and Biases in Society & Introduces the social foundations of bias by explaining how stereotypes form, persist, and shape expectations about gender roles. Necessary to understand why certain translation outputs may reflect or reinforce societal gender norms. \\

Machine and Algorithmic Bias & Explains how social biases can enter machine learning systems through data, design choices, or feedback mechanisms. Sets the groundwork for understanding how gender bias can emerge in translation models used in this thesis. \\

Gender Bias in English-German Translation & Focuses on the specific challenges of translating between English and German, where the lack of grammatical gender in English and its necessity in German can cause biased outputs. Defines the types of gender bias relevant to the classification task in this thesis. \\

Bias Detection Frameworks & Reviews existing methods for identifying gender bias in language data and ML outputs. Helps justify the choice of a classification approach for detecting bias in translations. \\

\bottomrule
\end{tabularx}
\caption{Key concepts relevant to this thesis}
\label{tab:key-concepts}
\end{table}

\subsection{Citation Tracking}
Backward citation searching involved reviewing references cited by selected papers, prioritizing frequently cited and foundational works relevant to gender bias in MT. Forward citation searching used Google Scholar's "cited by" function to identify newer research citing those key papers. Filtering with specific terms (e.g., "German," "machine translation") was applied during forward search to maintain focus.

\subsection{Selection Criteria and Screening Process}
Titles and abstracts were manually screened to select relevant studies. Inclusion criteria required sources to specifically address gender bias in MT, provide examples or discussions of gender-related errors, or explain the significance of gender bias in this context. Exclusion criteria filtered out studies focusing on general NLP bias without a direct link to MT, non-gender biases without clear gender connection, and highly technical papers lacking contribution to the general understanding of gender bias. Full texts were reviewed after initial screening to confirm relevance and extract insights. Redundant sources not providing new perspectives aligned with the thesis goals were excluded.

\subsection{Management of Literature}
The Zotero reference manager was used continuously to collect and organize papers. As of this thesis, 34 sources have been included, while initial searches yielded over 18,000 results.

\section{Stereotypes and Biases in Society}
\section{Machine and Algorithmic Bias}
\section{Gender Bias in English-German Translation}
\section{Bias Detection Frameworks}


% ### 2.2 Gender Bias in Machine Translation

% ### 2.2.1 Existence of Gender Bias

% - Explain that gender bias in MT is well-documented.
    
%     **Use:** Savoldi et al. (2021), Prates et al. (2019), Stanovsky et al.
    

% ### 2.2.2 Empirical Evidence

% - Summarize how researchers measured bias (benchmarks, manual annotation, templates).
    
%     **Use:** Benchmark papers and empirical studies.
    

% ### 2.3 Bias Detection Methods

% - Overview of rule-based, ML-based, and LLM-based approaches.
    
%     **Use:** Bias detection in NLP literature, classifier-based and prompt-based approaches.
    

% ### 2.4 Gender Bias in English-German Translation

% - Specific challenges in German (grammatical gender).
% - Mention corpora used (Wikipedia Biographies, etc.).
    
%     **Use:** Papers dealing with German ↔ English MT and gender.
    

% ### 2.5 Positioning Your Work

% - How your work builds on these findings.
% - What gap you address (binary classifier + demo).
    
%     **Use:** Comparison with prior methods.