\chapter{Related Work}

This section outlines key findings of related work on gender bias in MT, with a focus on the English-German language pair. The research aims are to (1) define the core concept of gender bias in MT, (2) establish the relevance of the topic, (3) identify the research gap, and (4) justify technical design choices. To support this, I examine datasets, model types, and tools used in previous studies.

This review only partially followed a systematic literature review methodology, given the large volume of literature on gender bias in MT. Instead, it employed an iterative and selective search process based on my four specified research goals to remain practical within the scope and timeline of this work. The qualitative Information Systems literature review framework developed by \citet{schryenWritingQualitativeLiterature2015} was used to frame, synthesize and interpret this review. It was then further guided by the findings of \citet{shresthaExploringGenderBiases2022}, who have already conducted a systematic literature review on gender bias in ML and AI. These references helped identify key topic areas, define relevant keywords and thematic categories, and supported the positioning and justification of this thesis’s methodological choices.

\section{Literature Search Process}

\subsection{Search Sources and Tools}
Sources were primarily searched on \href{https://scholar.google.com/}{Google Scholar} and \href{https://www.perplexity.ai/}{Perplexity}, which served as an additional search engine. Prompts and outputs from Perplexity have been saved and are included in the appendix. To organize and manage the collected sources, \href{https://www.zotero.org/}{Zotero} was used throughout the process.

\subsection{Search Terms and Time Frame}
Key search terms included \textit{gender bias}, \textit{machine translation}, \textit{AI}, \textit{machine learning}, \textit{German}, and \textit{stereotypes}. The focus was on literature published between 2019 and 2025 to maintain relevance and currency, while foundational and definitional works from earlier periods were selectively included.

\subsection{Citation Tracking}
Backward citation searching involved reviewing references cited by selected papers, prioritizing frequently cited and foundational works relevant to gender bias in MT. Forward citation searching used Google Scholar's "cited by" function to identify newer research citing those key papers. Filtering with specific terms (e.g., "German," "machine translation") was applied during forward search to maintain focus.

\subsection{Selection Criteria and Screening Process}
Titles and abstracts were manually screened to select relevant studies. Inclusion criteria required sources to specifically address gender bias in MT, provide examples or discussions of gender-related errors, or explain the significance of gender bias in this context. Exclusion criteria filtered out studies focusing on general NLP bias without a direct link to MT, non-gender biases without clear gender connection, and highly technical papers lacking contribution to the general understanding of gender bias. Full texts were reviewed after initial screening to confirm relevance and extract insights. Redundant sources not providing new perspectives aligned with the thesis goals were excluded.

\subsection{Management of Literature}
The Zotero reference manager was used continuously to collect and organize papers. As of this thesis, 34 sources have been included, while initial searches yielded over 18,000 results.


\section{Gender Bias in Machine Translation}

% ### 2.2 Gender Bias in Machine Translation

% ### 2.2.1 Existence of Gender Bias

% - Explain that gender bias in MT is well-documented.
    
%     **Use:** Savoldi et al. (2021), Prates et al. (2019), Stanovsky et al.
    

% ### 2.2.2 Empirical Evidence

% - Summarize how researchers measured bias (benchmarks, manual annotation, templates).
    
%     **Use:** Benchmark papers and empirical studies.
    

% ### 2.3 Bias Detection Methods

% - Overview of rule-based, ML-based, and LLM-based approaches.
    
%     **Use:** Bias detection in NLP literature, classifier-based and prompt-based approaches.
    

% ### 2.4 Gender Bias in English-German Translation

% - Specific challenges in German (grammatical gender).
% - Mention corpora used (Wikipedia Biographies, etc.).
    
%     **Use:** Papers dealing with German ↔ English MT and gender.
    

% ### 2.5 Positioning Your Work

% - How your work builds on these findings.
% - What gap you address (binary classifier + demo).
    
%     **Use:** Comparison with prior methods.