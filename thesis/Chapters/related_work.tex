\chapter{Related Work}

This section outlines key findings of related work on gender bias in MT, with a focus on the English-German language pair. The aim is to (1) define the core concept of gender bias in MT, (2) establish the relevance of the topic, (3) identify the research gap, and (4) justify technical design choices. To support this, I examine datasets, model types, and tools used in previous studies.

This literature review was structured according to the framework for qualitative Information Systems literature reviews developed by \citet{smacchiaDoesAIReflect2024} and further guided by the findings of \citet{shresthaExploringGenderBiases2022}, who conducted a systematic literature review on gender bias in ML and AI. It served as a reference point for identifying key areas of the topic and provided a basis for defining relevant keywords and thematic categories. This process helped identify empirical findings and technical approaches relevant to this thesis and supported the positioning and justification of its methodological choices.


\section{Literature Sourcing Methodology}



Since Machine Learning (ML) is a rapidly evolving field and older studies may be outdated, only
publications from 2015 onward are considered. Additionally, only publicly available English sources
are included. Literature was primarily found using keyword-based searches on Google Scholar. Initial
search terms such as "gender bias" and "machine translation" yielded over 18,000 results. 

- refer to 03 of methodology

- Define the review's purpose: synthesis, interpretation, and guidance for future research.
- Describe databases used (ACL, Google Scholar), search terms, filtering process.
- Establish inclusion and exclusion criteria for selecting literature.
- Utilize qualitative data analysis tools (e.g., NVivo) for coding and thematic analysis.
- Ensure transparency and replicability in the review process.

\section{Gender Bias in Machine Translation}

% ### 2.2 Gender Bias in Machine Translation

% ### 2.2.1 Existence of Gender Bias

% - Explain that gender bias in MT is well-documented.
    
%     **Use:** Savoldi et al. (2021), Prates et al. (2019), Stanovsky et al.
    

% ### 2.2.2 Empirical Evidence

% - Summarize how researchers measured bias (benchmarks, manual annotation, templates).
    
%     **Use:** Benchmark papers and empirical studies.
    

% ### 2.3 Bias Detection Methods

% - Overview of rule-based, ML-based, and LLM-based approaches.
    
%     **Use:** Bias detection in NLP literature, classifier-based and prompt-based approaches.
    

% ### 2.4 Gender Bias in English-German Translation

% - Specific challenges in German (grammatical gender).
% - Mention corpora used (Wikipedia Biographies, etc.).
    
%     **Use:** Papers dealing with German ↔ English MT and gender.
    

% ### 2.5 Positioning Your Work

% - How your work builds on these findings.
% - What gap you address (binary classifier + demo).
    
%     **Use:** Comparison with prior methods.